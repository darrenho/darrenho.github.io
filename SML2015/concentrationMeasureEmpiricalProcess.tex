\documentclass[12pt]{beamer}
%\usepackage[usenames,dvipsnames]{xcolor}

\usepackage{_defsAndPackages675notation}
\usepackage{_defsAndPackages675beamer}

%\DeclareMathSizes{12}{12}{5}{12}
\newcommand{\parenthetical}[2]{#1  \scriptstyle \alr{( #2)}}
\date{}

\begin{document}

\title{\alg{Some theoretical tools}}
\subtitle{\classTitle}

\begin{frame}
\maketitle
%\titlepage
%\begin{figure}[h!]
%  \centering
%  \includegraphics[width=1in]{.../figures/CSU_logo2.eps}
%\end{figure}
%
\organization
%
\end{frame}

\transitionSlide{Normal means}

\begin{frame}
\frametitle{A simpler model}
Suppose that $Y \sim (\mu,1)$ and let 
\[
L_q(\mu) = 2^{q-2}(Y-\mu)^2 + \lambda |\mu|^q
\]
and
\[
\hat{\mu}_q = \argmin_{\mu} L_q(\mu)
\]

Then,

\begin{itemize}
\item \alg{$q = 0$} $\Rightarrow \hat{\mu}_0  = Y$  
\item \alg{$q = 2$} $\Rightarrow \hat{\mu}_2  = Y/(\lambda + 1)$  
\item \alg{$q= 1$}?
\end{itemize}
\end{frame}

\begin{frame}
\frametitle{Subdifferential}
To theoretically solve this optimization problem, we use the notion of a \alg{subderivative}.

\vsp
We call $c$ a subderivative of $f$ at $X_0$ provided
\[
f(X) - f(X_0) \geq  c(X - X_0)
\]

A convex function can be optimized by setting the subderivative $= 0$

\vsp
The \alg{subdifferential} $\partial f|_{X_0}$ is the set of subderivatives. 
\vsp

$X_0$ minimizes $f$ if and only if $0 \in \partial f|_{X_0}$.

\end{frame}

\begin{frame}
\frametitle{Subdifferential in action}
For $\rho(\mu) = |\mu|$,
\[
\partial \rho|_{\mu}
=
\begin{cases}
\{ -1\} & \textrm{ if } \mu < 0 \\
[-1,1] & \textrm{ if } \mu = 0 \\
\{ 1\} & \textrm{ if } \mu > 0 
\end{cases}
\]
Therefore
\[
\partial L_1|_{\mu}
=
\begin{cases}
\{ \mu - Y - \lambda\} & \textrm{ if } \mu < 0 \\
\{\mu - Y + \lambda z : -1 \leq z \leq 1 \} & \textrm{ if } \mu = 0 \\
\{ \mu - Y + \lambda\} & \textrm{ if } \mu > 0 \\
\end{cases}
\]

\end{frame}

\begin{frame}
\frametitle{$\ell_1$ and soft-thresholding}

$\hat\mu_1$ minimizes $L_1$ if and only if $0 \in \partial L_1|_{\hat\mu_1}$
\vsp

So..
\[
\hat{\mu}_1 
=
\begin{cases}
Y + \lambda & \textrm{ if } Y < -\lambda \\
0 & \textrm{ if } -\lambda \leq Y \leq \lambda \\
Y - \lambda & \textrm{ if } Y > \lambda 
\end{cases}
\]

This can be written
\[
\hat{\mu}_1 = \textrm{sgn}(Y)(|Y| - \lambda)_+
\]

This is known as \alg{soft thresholding}
\end{frame}

\begin{frame}
\frametitle{Orthogonal design: Example}
Suppose now that ($p \leq n$)
\[
Y = \X\beta + \epsilon,
\]
where $\X^{\top} \X/n = I$.   

\vsp
Let's solve 
\[
\hat\beta_{\lambda} = \argmin_{\beta} \frac{1}{2n}|| \X \beta - Y||_2^2 + \lambda ||\beta||_1
\]
%Note that $\hat\beta_{LS} = \X^{\top}Y/n$ and
\[
\frac{1}{2n}|| \X \beta - Y||_2^2 
\propto 
\frac{\beta^{\top}\X^{\top}\X\beta}{2n} - \frac{\beta^{\top}\X^{\top}Y}{n} 
=
\frac{\beta^{\top}\beta}{2} - \beta^{\top}\hat\beta_{LS}
\]
Now,
\[
\frac{1}{2n}|| \X \beta - Y||_2^2 + \lambda ||\beta||_1 
= 
\sum_{j=1}^p \left( \beta_j^2/2 - \beta_j \hat{\beta}_{LS,j} +\lambda |\beta_j|\right)
\]
\end{frame}

\begin{frame}
\frametitle{Orthogonal design}
We can minimize this component wise: 
\[
L(\beta) =  \beta^2/2 - \beta \hat{\beta}_{LS} +\lambda |\beta| \parenthetical{\qquad}{\textrm{dropping the } j}
\]
\vsp

This can be optimized using \alo{subdifferentials} in exactly the same way


\script{I'll leave this as an \smallCapGreen{Exercise}}

\vsp
This results in \alo{soft-thresholding} the least squares solution.  

\vsp
This rationale can be extended to make the lasso \alo{gradient (coordinate) descent} explicit

\script{And is the backbone of \alr{glmnet}}
\end{frame}
%
%\begin{frame}
%\frametitle{From orthogonal to non-orthogonal design}
%\smallCapGreen{Notation:}
%We will refer to the covariates via $x_j$, observations via $X_i$, and the $j^{th}$ covariate of the
%$i^{th}$ observation as $X_{ij}$.
%
%\vvsp
%An iterative algorithm for finding $\hat\beta \equiv \hat\beta_{\lambda}$ is:
%
%\vsp
%Set $\hat \beta = (0,\ldots,0)^{\top}$.  Then for
%$j = 1,\ldots,p$:
%\begin{enumerate}
%\item Define $R_i = \sum_{k \neq j} \hat \beta_{k} X_{ik}$
%\item Form $\hat \beta_j$ by simple least squares of $(R_i)_{i=1}^n$ on $x_j$
%\item Soft-threshold these coefficients: $\hat\beta_j = \textrm{sgn}(\hat{\beta}_j)(|\hat{\beta}_j| - \lambda/||x_j||_2^2)_{+}$
%\end{enumerate}
%\script{This insight can be extended to sparse additive models as well, which we'll return to later.  \alr{Question:}
%Who knows what non-parametric regression is?}
%\end{frame}

\begin{frame}
\frametitle{Normal means}
Note that the orthogonal design linear model is an example of a \alg{normal means} problem:

\vsp
Let $\epsilon \sim N(0,I)$, then
\[
Y = \X\beta + \epsilon \Leftrightarrow W \stackrel{D}{=} \beta + \frac{1}{\sqrt{n}} \epsilon
\]

This turns out to be an even more powerful idea..
\end{frame}

\begin{frame}
\frametitle{Normal means}
Let
\begin{itemize}
\item $\mathcal{H}$ be a real, separable Hilbert space with inner product $\langle \cdot, \cdot \rangle$
\item $(\phi_i)$ be an orthonormal basis for $\mathcal{H}$
\end{itemize}
Then we can imagine a signal $h$ being observed with a white noise Gaussian \alo{process} 
\[
Y(t) = h(t) + \epsilon(t)
\]
\script{Technically, this doesn't exist.  Rather we can observe functionals $Y(t) dt = h(t) dt + d \epsilon(t)$}
\vsp

We make observations of this signal via inner products:
\[
y_i = \langle Y , \phi_i\rangle = \langle h + \epsilon , \phi_i\rangle = h_i+ \epsilon_i
\]

\vsp
As linear operations of Gaussians are Gaussians, $\epsilon_i \sim N(0,1)$ 
\end{frame}

\transitionSlide{Concentration inequalities and empirical processes}

\begin{frame}[fragile]
\frametitle{High level overview}
The core of modern machine learning theory rests with the following structure:

\vsp
%Let $(T,d)$ be a metric space.  Define a $\R^q$-valued process (generally, $q = 1$) $(X_t)_{t\in T}$.  Then
%for all $u$ sufficiently large, we want to find bounds of the form
%\[
%\P(\norm{X_t - X_s}_q > u) \leq C\exp\left\{-cf\left(\frac{u}{d(s,t)} \right)\right\}
%\]
%This is known as an \alg{increment condition}
\begin{enumerate}
\item \smallCapGreen{Concentration inequalities:} Show that a random quantity is close to its mean with 
high probability
\begin{enumerate}
\item Hoeffding's
\item McDiarmid's
\item Bernstein's
\end{enumerate}
\item \smallCapGreen{Uniform bounds:} Guarantee that a set of random quantities are all simultaneously close 
to their means with high probability
\begin{enumerate}
\item VC-dimension
\item Rademacher complexity
\item Covering/bracketing numbers
\end{enumerate}

\end{enumerate}
\end{frame}
%
%\begin{frame}[fragile]
%\frametitle{High level overview}
%The main tools that will be used are
%\begin{itemize}
%\item \smallCapGreen{Concentration of measure:} Gives a non-asymptotic bound on tail probability of a stochastic process
%\item \smallCapGreen{Generic chaining:} Related to the idea of compactness, we can `chain' together bounds 
%on subsets of $T$.  In particular, we can form bounds for $\E\sup_{t \in T} X_t$
%\end{itemize}
%\end{frame}

\begin{frame}[fragile]
\frametitle{Motivation}
\smallCapGreen{Goal:} (concentration inequalities) $+$ (complexity measure) $=$ uniform coverage
of a stochastic process (i.e. $\sup_{t \in T} X_t$). 
\vsp

 When would this be useful?

\vsp
Suppose we have data $\data$ and a loss function $\ell_f$ and we wish to find a function $\hat{f}$ that can predict
a new $Y$ from an $X$

\vsp
Form the excess risk
\[
\mathcal{E}(\hat f) = \P\ell_{\hat{f}} - \inf_{f \in\F} \P\ell_{f}
\]
and $\hat{f} = \argmin_{f\in\F} \hat{\P} \ell_f$
\end{frame}

\begin{frame}[fragile]
\frametitle{Recall}

 $\hat{\P} = n^{-1} \sum_{i=1}^n \delta_{X_i}$ is the empirical measure.  
 
 \vsp This can be
interpreted in two ways:

\begin{itemize}
\item \smallCapGreen{Expectation:} Let $f$ be a function, then we write
\[
\hat\P f = \int f \, d\hat\P = \frac{1}{n}\sum_{i=1}^n f(X_i)
\]
\item \smallCapGreen{Measure:} Let $C$ be a (measurable) \alo{set}, then we write
\[
\hat\P C = \int \mathbf{1}_C d\hat\P = \frac{1}{n}|\{i: X_i \in C\}|
\]
\end{itemize}

\script{These notions are used interchangeably, and motivate using $\P$ for both
probability and expectation}
\end{frame}

\begin{frame}[fragile]
\frametitle{Back to motivation}
Apply the \smallCapGreen{$2-\epsilon$ technique:}
\begin{align*}
\mathcal{E}(\hat f) &  = \P\ell_{\hat{f}} - \hat\P\ell_{\hat{f}} + \hat\P\ell_{\hat{f}} - \inf_{f \in\F} \P\ell_{f} \\
& \leq \P\ell_{\hat{f}} - \hat\P\ell_{\hat{f}} + \hat\P\ell_{f_*} - \P\ell_{f_*} \\
& \leq 2\sup_{f\in\F} |\P\ell_{f} - \hat\P\ell_{f}|
\end{align*}
Where $f_*$ is such that $ \P\ell_{f_*} = \inf_{f \in\F} \P\ell_{f}$

\vsp
So, fixing an $\epsilon > 0$
\begin{align*}
\P(\mathcal{E}(\hat f) > 2\epsilon )   
& \leq 
\P( | \sup_{f \in \F} | (\hat \P - \P)\ell_f | > \epsilon) \\
& \leq
\frac{\E \sup_{f \in \F} |(\hat\P - \P)\ell_f|}{\epsilon}
\end{align*}
\end{frame}

\begin{frame}[fragile]
\frametitle{Motivation}
\smallCapGreen{Conclusion:}
\[
\P(\mathcal{E}(\hat f) > 2\epsilon )    \leq
\epsilon^{-1}\E \sup_{f \in \F} |(\hat\P - \P)\ell_f|
=
\epsilon^{-1}\E \norm{\hat\P - \P}_\F
\]
We can bound the excess risk of an estimator $\hat f$ by bounding the supremum of the difference between
the empirical measure and true measure

\vsp
Note that:
\begin{itemize}
\item Using the previous notation, $X_t = (\hat \P - \P)\ell_f$, and $T = \F$

\script{Sometimes the index set is considered $\mathcal{L} = \{ \ell_f : f\in \F\}$}
\item The stochastic process $\mathbb{G} = \sqrt{n}(\hat \P - \P)$ is the \alg{empirical process}
\end{itemize}
\end{frame}

\begin{frame}[fragile]
\frametitle{Empirical process}
The stochastic process $(\hat \P - \P)\ell_f$ is zero mean and hence we know by the SLLN that for all $f \in \F$
\[
(\hat \P - \P)\ell_f \rightarrow 0 \textrm{ a.s}
\]
\script{Assuming $\P \ell_f$ exists, of course}

\vsp
However, this doesn't give us \alo{uniform} control

\script{i.e: this doesn't imply that the supremum goes to zero}

\vsp
We call an index set $\F$ a \alg{Glivenko-Cantelli class} if
\[
\sup_{f \in \F} |( \hat \P - \P)\ell_f| = \norm{\hat\P - \P}_\F \rightarrow 0 \textrm{ a.s}
\]

\end{frame}

\begin{frame}[fragile]
\frametitle{Glivenko-Cantelli: example}
A classical example is the \alo{empirical CDF}
\[
F_n(t) = \frac{1}{n} \sum_{i=1}^n \mathbf{1}_{(-\infty,t]}(X_i) = \hat\P f_t
\]
where $f_t(x) = \mathbf{1}_{(-\infty,t]}(x)$

\vsp
Often, we are attempting to estimate a \alg{functional} of the true CDF with a plug-in version using the 
empirical CDF

\script{True CDF: $F(t) = \P(X \leq t)$}

\vsp
\smallCapGreen{Sub Example:} Let $\theta = \theta(\P)$ given by the \alo{median}:
$\theta = \theta(\P)$ is argmin of $\P(-\infty,x] = \inf_x F(x)$ subject to $F(x) \geq 1/2$
\vsp

Then, we might estimate $\theta(\P)$ with 
$\hat\theta = \theta(\hat\P)$ by plugging in $F_n$
\end{frame}

\begin{frame}[fragile]
\frametitle{Glivenko-Cantelli: example}
The \alg{Glivenko-Cantelli theorem} says that 
\[
\sup_{t \in \R} |F_n(t) - F(t)| \rightarrow 0 \textrm{ a.s.}
\]
If we write  $\F = \{f_t:  f_t(x) = \mathbf{1}_{(-\infty,t]}(x), t \in \R\}$, then
\[
\sup_{t \in \R} |F_n(t) - F(t)| = \norm{F_n - F}_{\R} = \norm{\hat \P - \P}_\F
\]
and hence $\F$ is a Glivenko-Cantelli (G.C.) class.
\end{frame}

\begin{frame}[fragile]
\frametitle{Glivenko-Cantelli: example}
\script{Technical condition: $\P(-\infty,t] > 1/2$ for each $t > \theta(\P)$.  This forces a \alo{continuity} property
that $|$median($\P$) - median($\P$')$| < \epsilon$ if $\P$ and $\P$' are uniformly close.}
\vsp

\smallCapGreen{Sub Example:} As $\F$ is G.C., for all $\delta > 0$, for $n$ large enough
\[
\sup_t | \hat\P(-\infty,t] - \P(-\infty,t]| < \delta
\]
Fix $\epsilon > 0$. Choose $\delta$ such that 
\begin{align*}
\P(-\infty,\theta-\epsilon] & < \frac{1}{2} - \delta \parenthetical{\quad}{\textrm{This is always possible}}\\
\P(-\infty,\theta+\epsilon] & > \frac{1}{2} + \delta \parenthetical{\quad}{\textrm{This requires condition}}
\end{align*}
\end{frame}

\begin{frame}[fragile]
\frametitle{Glivenko-Cantelli: example}
Now, 
\[
\P(-\infty,\hat\theta] > \underbrace{\hat\P(-\infty,\hat\theta] - \delta}_{\textrm{uniform closeness}} \geq 1/2 - \delta
\]
Hence, $\hat\theta > \theta - \epsilon$ as BWOC: 
\[
\hat\theta \leq \theta - \epsilon \Rightarrow\P(-\infty,\hat\theta]  \leq\P(-\infty,\theta - \epsilon] < 1/2 - \delta
\]
\vsp

%Also, $\forall \theta' < \hat\theta$
%\[
%\P(-\infty,\theta'] < \hat\P(-\infty,\theta'] + \delta \leq 1/2 + \delta
%\]
%
%\[
%\theta' < \hat\theta \Rightarrow \theta' < \theta + \epsilon \Rightarrow \hat\theta \leq \theta + \epsilon
%\]
%\[
%\P(-\infty,\hat\theta] > \hat\P(-\infty,\hat\theta] - \delta \geq 1/2 - \delta
%\] 
Also, it can be shown that $\hat\theta \leq \theta + \epsilon$

\vsp
Hence, uniform closeness of $F_n$ to $F$ shows that the sample and populations medians are close


\script{Note, we have asked for much more than needed, sometimes this can be too much}
\end{frame}

\begin{frame}[fragile]
\frametitle{Glivenko-Cantelli: example}
This gets refined to a rate of convergence by the Dvoretzky-Kiefer-Wolfowitz (DKW) inequality 
\[
\P( \norm{F_n - F}_{\infty} > \epsilon) \leq 2e^{-2n\epsilon^2}
\]
\script{Both the constants 2 cannot be improved upon (Massart (1990))}

\vsp
This result, along with the previous discussion gets us a rate of convergence for the median
\end{frame}

\transitionSlide{Some lasso theory}


\begin{frame}
\frametitle{Low assumption prediction: the lasso}
\alo{Remember:}  Prediction risk
  \[
  R(\beta) 
  = 
  \mathbb{E}_Z \left[\left(Y - X^\top \beta\right)^2 \right]
  =
  \mathbb{E}_Z \left[\left(Y - X^\top \beta\right)^2 | \mathcal{D}\right]
  \]

  \vsp
  Define the oracle estimator
  \[
  \beta_t^* = \argmin_{\{\beta:||\beta||_1 \leq t\}} R(\beta)
  \]
\script{\alr{Important:} This does not assume that $\E Y|X$ is linear in $X$!}

  \vsp
  The \alg{excess risk} is
  \[
  \mathcal{E}(\hat\beta_t,\beta_t^*) = R(\hat\beta_t) - R(\beta_t^*)
  \]
\end{frame}

\begin{frame}
  \frametitle{Persistence}
 
  A procedure is \alg{persistent} for a set of measures $\mathcal{P}$ if 
  \[
\forall \P \in \mathcal{P}, \qquad  \mathcal{E}(\hat\beta_t,\beta_t^*) \stackrel{p}{\rightarrow} 0
  \]
\script{This is convergence in probability.  What is \alo{random}?}
  \vsp
%  \script{Greenshtein, Ritov (2004)} 

Define the following set of distributions on $\R \times \R^p$: Let $C_\mathcal{P} < \infty$ and 
\[
\mathcal{P} = \{ \P: \P Y^2 < C_\mathcal{P}\textrm{, and } |x_j| < C_\mathcal{P} \textrm{ almost surely, } j = 1,\ldots,p\}
\]

\vsp
We'd like to know how fast $t$ can grow while still maintaining persistency. 

\script{Note: this set $\mathcal{P}$ is more restrictive than needed}
\end{frame}

\begin{frame}
  \frametitle{Useful results and observations}
Let 
\begin{itemize}
\item $Z = (x_0, x_1, \ldots, x_p)^{\top}$, where $x_0 = Y$
\item $\gamma = (-1,\beta_1,\ldots,\beta_p)^{\top}$
\end{itemize}
Then, for $\ell_\beta(Z) = (Y - X^{\top}\beta)^2$,
\[
\P \ell_{\beta} = \P(Y - X^{\top}\beta)^2 = \gamma^{\top} \Sigma \gamma,
\]
where $\Sigma_{jk} = \P Z_j Z_k$ for $0 \leq j,k \leq p$

\vsp
Likewise, 
\[
\hat{\P}\ell_{\beta} = \gamma^{\top} \hat\Sigma \gamma,
\]
where $\hat\Sigma_{jk} = n^{-1} \sum_{i = 1}^n Z_{ij}Z_{ik}$ for $0 \leq j,k \leq p$

\script{These can be written: $\Sigma = \P ZZ^{\top}$ and $\hat\Sigma = \hat\P ZZ^{\top}$}
\end{frame}

\begin{frame}
  \frametitle{Persistence theorem}
\begin{theorem}
Over any $\P$ in $\mathcal{P}$, the procedure
\[
\argmin_{\beta \in \{\beta: ||\beta||_1 \leq t\}} \hat{\P} \ell_\beta
\]
is persistent provided $\log p = o(n)$ and 
\[
t = t_n =  o\left( \left(\frac{n}{\log p} \right)^{1/4} \right)
\]
\end{theorem}
\script{This theorem appears in Greenshtein, Ritov (2004)}

\script{It's worth noting that this rate is improved to a 
square root in Bartlett, et al. (2012).  This is at the expense of higher order powers of the log terms.
These logarithmic powers could be removed by bounding 
Talagrand's $\gamma_2$ functional directly, instead of an entropy integral}
\end{frame}


\begin{frame}
  \frametitle{Deterministic asymptotic notation}
We write $a_n = O(b_n)$ (and say \alg{big ohh}) provided
\[
\frac{a_n}{b_n} = O(1),
\]
where 
\[
c_n = O(1) 
\]
means 
\begin{itemize}
\item There exists a $C$
\item Such that for sufficiently large $N$
\item For all $n \geq N$
\item $c_n \leq C$
\end{itemize}
\end{frame}

\begin{frame}
  \frametitle{Deterministic asymptotic notation}
We write $a_n = o(b_n)$ (and say \alg{little ohh}) provided
\[
\frac{a_n}{b_n} = o(1),
\]
where 
\[
c_n = o(1) 
\]
means 
\begin{itemize}
\item For all $\epsilon>0$
\item There exists an $N$
\item Such that for all $n \geq N$
\item $c_n \leq \epsilon$
\end{itemize}
\end{frame}

\begin{frame}
  \frametitle{Stochastic asymptotic notation}
We write $a_n = O_p(b_n)$ (and say \alg{big ohh p}) provided
\[
\frac{a_n}{b_n} = O_p(1),
\]
where 
\[
c_n = O_p(1) 
\]
means 
\begin{itemize}
\item For all $\delta$
\item There exists a C
\item Such that for sufficiently large $N$
\item For all $n \geq N$
\item $\P(|c_n|  \geq C)\leq \delta$
\end{itemize}
\script{This is also called \alg{bounded in probability}, and is related to convergence in distribution}
\end{frame}

\begin{frame}
  \frametitle{Stochastic asymptotic notation}
We write $a_n = o_p(b_n)$ (and say \alg{little ohh p}) provided
\[
\frac{a_n}{b_n} = o_p(1),
\]
where 
\[
c_n = o_p(1) 
\]
means 
\begin{itemize}
\item For all $\epsilon>0,\delta > 0$
\item There exists an $N$
\item Such that for all $n \geq N$
\item $\P(|c_n| \geq \epsilon) \leq \delta$
\end{itemize}
\end{frame}

\begin{frame}
  \frametitle{Stochastic asymptotic notation}
Note that if we have random variables $(X_n)$ and $X$, then
\[
X_n \rightarrow X \textrm{ in probability} \Leftrightarrow X_n - X = o_p(1)
\]
We can also express Slutsky's theorem(s)
\vsp

\begin{itemize}
\item $o_p(1) + O_p(1) = ?$
\item $o_p(1) O_p(1) = ?$
\item $o_p(1) + o_p(1)O_p(1) = ?$
\end{itemize}
\end{frame}

\begin{frame}
  \frametitle{Persistence proof}
  Note that, $\sup_{\beta \in \{b: \norm{b}_1 \leq t\}} ||\beta||_1 \leq t$.   Also, 
  \vsp
  
  \begin{lemma}
  \label{lem:quad-form} 
  Suppose $a \in \mathbb{R}^p$ and $A \in
  \mathbb{R}^{p \times p}$.  Then
  \begin{equation*} 
    a^\top A a 
    \leq \norm{a}^2_{1}
    \norm{A}_{\infty},
  \end{equation*}
  where $\norm{A}_\infty:=\max_{i,j} |A_{ij}|$ is the entry-wise max norm.
\end{lemma}
 \begin{proof} 
   \begin{align*}
     a^\top A a & \underbrace{\leq}_{\textrm{H{\"o}lder's}} \norm{a}_{1} \norm {A a}_{\infty} 
      \leq \norm{a}_{1} \max_{ij} |A_{ij}| \norm{a}_1  
      = \norm{a}^2_{1} \norm{A}_{\infty},
   \end{align*}
 \end{proof}
These facts imply..
  \end{frame}
  
  \begin{frame}
  \frametitle{Persistence proof}
  \begin{align}
  \mathcal{E}(\hat\beta_t,\beta_t^*) 
  & = 
  \underbrace{R(\hat\beta_t)}_{ \hat\gamma_t^{\top} \Sigma \hat\gamma_t } - 
  \underbrace{R(\beta_t^*)}_{ (\gamma_t^*)^{\top} \Sigma  (\gamma_t^*)} \\
  & = 
  \hat\gamma_t^{\top} \Sigma \hat\gamma_t - \hat\gamma_t^{\top} \hat\Sigma \hat\gamma_t 
  +  \hat\gamma_t^{\top} \hat\Sigma \hat\gamma_t -  (\gamma_t^*)^{\top} \Sigma  (\gamma_t^*) \\
  & \leq
  \hat\gamma_t^{\top} \Sigma \hat\gamma_t - \hat\gamma_t^{\top} \hat\Sigma \hat\gamma_t 
  +   (\gamma_t^*)^{\top} \hat\Sigma  \gamma_t^* -  (\gamma_t^*)^{\top} \Sigma  \gamma_t^*\\  
  & =
  \hat\gamma_t^{\top} (\Sigma - \hat\Sigma) \hat\gamma_t 
  +  (\gamma_t^*)^{\top}( \hat\Sigma  - \Sigma) (\gamma_t^*) \\
  & \leq
  2\sup_{\beta \in \{b : ||b||_1 \leq t\}} \gamma_t^{\top} (\Sigma - \hat\Sigma) \gamma_t  \parenthetical{\quad}{2\epsilon \textrm{ trick}}\\
  & \leq   
  2\sup_{\beta \in \{b : ||b||_1 \leq t\}} \norm{\gamma_t}_1^2 \norm{\Sigma - \hat\Sigma}_{\infty} 
  \parenthetical{\quad}{Lemma} \\
  & \leq   
  2(t+1)^2 \norm{\Sigma - \hat\Sigma}_{\infty}   
    \end{align}
Can we control the $\sup$-norm part?
\end{frame}


  \begin{frame}
  \frametitle{Persistence proof}
\smallCapGreen{Nemirovski's inequality:}  Let $\xi_i  \in \R^p$, $i=1,\ldots,n$ be independent, zero mean, finite variance
random variables with $p \geq 3$.  Define $S_n = \sum_{i=1}^n \xi_i$. 
Then for every $q \in [2,\infty]$
\[
\E ||S_n ||_q^2 \leq e(2\log(p)-1)  \sum_{i=1}^n \E ||\xi_i||_q^2
\]

\script{Juditsky, Nemirovski (2000), D\"umbgen, et al. (2010)}
\vsp

This should be compared with the na\"ive bound:
\[
\E ||S_n ||_q^2 \leq \sum_{i=1}^n\sum_{i'=1}^n \E ||\xi_i||_q||\xi_{i'}||_q
\]
\end{frame}
  \begin{frame}
  \frametitle{Persistence proof: Nemirovski's inequality}

\alo{Motivation:} Under Nemirovski's assumptions, 
\begin{itemize}
\item $\E S_n^2 = \sum_{i=1}^n \E\xi_i^2 \parenthetical{\quad}{p=1}$
\item In a Hilbert space with inner product $\langle \cdot, \cdot \rangle$
\[
\E ||S_n||^2 = \sum_{i,i'}^n \E \langle \xi_i, \xi_{i'} \rangle = \sum_{i=1}^n \E ||\xi_i||^2
\]
\item What about a Banach space (e.g. $||\cdot||_q, q \neq 2$)?
\end{itemize}
\end{frame}

  \begin{frame}
  \frametitle{Persistence proof}
\smallCapGreen{Nemirovski's inequality:}  Let $\xi_i  \in \R^p$, $i=1,\ldots,n$ be independent, zero mean, finite variance
random variables with $p \geq 3$.  Define $S_n = \sum_{i=1}^n \xi_i$. 
Then for every $q \in [2,\infty]$
\[
\E ||S_n ||_q^2 \leq e(2\log(p)-1)\sum_{i=1}^n \E ||\xi_i||_q^2
\]

\vsp
Let $\xi_i = \textrm{vec}\left(\frac{1}{n} \left(Z_{ij}Z_{ik} - \E Z_{j}Z_{k}\right) \right) \in \R^{(p+1)^2}$ be the vectorized
difference of \alo{empirical covariance} and the \alo{true covariance}

\vsp
Then
\[
\norm{\Sigma - \hat\Sigma}_{\infty}   = \norm{ \sum_{i=1}^n \xi_i}_\infty
\]
\end{frame}


  \begin{frame}
  \frametitle{Persistence proof}

\[
\E ||S_n ||_q^2 \leq e(2\log(p)-1) \sum_{i=1}^n \E ||\xi_i||_q^2
\]
\script{\smallCapGreen{Nemirovski's inequality}}

\begin{align*}
\left(\E\norm{\Sigma - \hat\Sigma}_{\infty} \right)^2
& \leq
\E\norm{\Sigma - \hat\Sigma}_{\infty}^2  \parenthetical{\qquad}{\textrm{Jensen's inequality}}\\
 & = 
\E \norm{ \sum_{i=1}^n \xi_i}_\infty^2 \\
 & \leq C \log( (p+1)^2)
\sum_{i=1}^n \E || \xi_i||_\infty^2 \\
 & \leq 4CC_{\mathcal{P}}^2 \log( p+1 ) \frac{1}{n} \parenthetical{\quad}{\P \in \mathcal{P}}\\
 & \lesssim \frac{\log(p)}{n}
\end{align*}
\end{frame}

  \begin{frame}
  \frametitle{Persistence proof: Conclusion}
  \begin{align}
\P\left(   \mathcal{E}(\hat\beta_t,\beta_t^*)  > \delta\right)
  & \leq
  \E[\mathcal{E}(\hat\beta_t,\beta_t^*) ] \delta^{-1} \parenthetical{\quad}{\textrm{Markov's inequality}}\\
& \leq   
  2\delta^{-1}(t+1)^2 \E \norm{\Sigma - \hat\Sigma}_{\infty}   \\
  & \lesssim
  2\delta^{-1}(t+1)^2 \sqrt{\frac{\log p}{n}}
    \end{align}
Therefore, we have \alo{persistence} provided $\log p = o(n)$ and
\[
t_n = o\left( \left(\frac{n}{\log p} \right)^{1/4} \right)
\]
Alternatively
\[
\mathcal{E}(\hat\beta_t,\beta_t^*) = O_p\left( t^2\sqrt{\frac{\log(p)}{n}}\right)
\]

\end{frame}

  \begin{frame}
  \frametitle{Probably approximately correct (PAC)}
Probability bound $\Leftrightarrow$ \alo{high probability} upper bound:
\[
\P(\textrm{error} > \delta) \leq \epsilon
\]
gets converted to: with probability $1-\epsilon$
\[
\textrm{error} \leq \delta
\]
\script{This is known as a PAC bound}

\vsp
\smallCapGreen{Example:} 
\[
\P(|\overline{X} - \mu| > \delta ) \leq \frac{\E(\overline{X} - \mu)^2}{\delta^2}
\]
Hence, with probability at least $1 - \frac{\sigma^2}{n\delta^2}$
\[
|\overline{X} - \mu| \leq \delta
\]
\end{frame}

  \begin{frame}
  \frametitle{Lasso theory: Summary}
It is important to note that we do \alo{not} assume...
\begin{itemize}
\item a linear model
\item an additive stochastic component (let alone, Gaussian errors)
\item that the design is `almost' uncorrelated
\end{itemize}
and we get that, with probability at least $1-C\delta^{-1}t^2\sqrt{\frac{\log(p)}{n}}$,
\[
R(\hat\beta_t) \leq R(\beta_t^*) + \delta
\]
\end{frame}

\transitionSlide{Orlicz norms}

\begin{frame}[fragile]
\frametitle{Orlicz norms}
As alluded to previously, this type of theory is driven by the \alo{tails} of the distribution
This is most commonly phrased in terms of \alg{Orlicz norms}

\vsp
Let $\psi$ be a non-decreasing, convex function such that $\psi(0) = 0$. Then:
\[
\norm{X}_\psi = \inf\{ C > 0: \E \psi\left(  \frac{|X|}{C}\right) \leq 1\}
\]
\script{Jensen's inequality shows this is a norm}

\vsp
There are two main cases
\begin{itemize}
\item \smallCapGreen{$L_p$ norm:} $\psi(x) = x^p \Rightarrow \norm{X}_\psi = \norm{X}_p = (\E |X|^p)^{1/p}$
\item \smallCapGreen{$p$-Orlicz:} $\psi_p(x)  = e^{x^p} - 1$ 
\end{itemize}
\end{frame}

\begin{frame}[fragile]
\frametitle{Orlicz norms}
Two important facts:
\begin{itemize}
\item $\norm{X}_{\psi_p} \leq \norm{X}_{\psi_q}(\log 2)^{1/q - 1/p}$, for $p \leq q$
\item $\norm{X}_{p} \leq p!\norm{X}_{\psi_1}$
\end{itemize}
\script{This allows us to interchange results about various norms, as long as we don't care about constants}
\end{frame}

\begin{frame}[fragile]
\frametitle{Orlicz norms}
By Markov's inequality
\begin{align*}
\P(|X| > x) & \leq \frac{\E \psi(|X|)/\norm{X}_{\psi}}{\psi(x)/\norm{X}_{\psi}} \\
& \leq
\frac{1}{\psi(x)/\norm{X}_{\psi}} \\
& =
\begin{cases}
\norm{X}_px^{-p} & \textrm{ if } \psi(x) = x^p \\
\frac{1}{e^{(x/\norm{X}_{\psi})^p}-1}  \asymp e^{-(x/\norm{X}_{\psi})^{p}}&  \textrm{ if } \psi(x) = \psi_p(x)
\end{cases}
\end{align*}
Hence, Orlicz norms allow us to encode the tail behavior of a random variable

\vsp
In fact, it works as an if and only if:

\vsp
If $\P(|X| > x) \leq C e^{-cx^p}$ then $\norm{X}_{\psi_p} \leq ( (1 + C)c^{-1})^{1/p} < \infty$
\end{frame}

\transitionSlide{Concentration inequalities}

\begin{frame}[fragile]
\frametitle{General form}
For showing results about empirical processes or performance guarantees for algorithms, we want results of the form
\[
\P\left(|f(Z_1,\ldots,Z_n) - \mu_n(f)| > \epsilon\right) < \delta_n
\]
where $\delta_n\rightarrow 0$ and $\mu_n(f) = \E f(Z_1,\ldots,Z_n)$.

\vsp
For statistical learning theory, we need \alo{uniform} bounds
\[
\P\left(\sup_{f \in \F} |f(Z_1,\ldots,Z_n) - \mu_n(f)| > \epsilon\right) < \delta_n
\]

\end{frame}

\begin{frame}[fragile]
\frametitle{Hoeffding's inequality}
Suppose $\mu = \E Z < \infty$ and $\P(Z \geq 0) = 1$.  Then for any $\epsilon > 0$
\[
\E Z = \int_0^\infty Z d\P \geq \int_\epsilon^\infty Z d\P \geq \epsilon \int_\epsilon^\infty  d\P = \epsilon \P(Z > \epsilon)
\]
Yielding \alg{Markov's inequality}

\vsp
This can be transformed to \alg{Chebyshev's inequality} by using the variance
\[
\P(|Z - \mu| > \epsilon) \leq \frac{\sigma^2}{\epsilon^2} \Rightarrow \P(|\overline{Z} - \mu| > \epsilon) \leq \frac{\sigma^2}{n\epsilon^2}
\]

\smallCapGreen{Observation:} This is nice, but does not decay exponentially fast.  However, it only makes a second moment
assumption
\end{frame}

\begin{frame}[fragile]
\frametitle{Hoeffding's inequality}
A different transformation occurs via a \alg{Chernoff bound}.  For any $t > 0$
\[
\P(Z > \epsilon) = \P\left( e^{tZ} > e^{t\epsilon}\right) \leq e^{-t\epsilon}\E[e^{tZ}]
\]
\script{This is the moment generating function.  Here we see increasing moment conditions giving tighter bounds}

\vsp
This can be minimized over $t$ as it is arbitrary
\[
\P(Z > \epsilon)  \leq \inf_{t > 0} e^{-t\epsilon}\E[e^{tZ}]
\]
\end{frame}

\begin{frame}[fragile]
\frametitle{Hoeffding's inequality}
This is the main content of the paper Hoeffding (1963)

\vsp
\smallCapGreen{Hoeffding's lemma:}
 Suppose $Z \in [a,b]$, then for any $t$
\[
\E[e^{tZ}] \leq e^{t\mu + t^2(b-a)^2/8}
\]

\vsp
\smallCapGreen{Proof idea:} 
\[
\E e^{tZ} \leq -\frac{a}{b-a}e^{tb} + \frac{b}{b-a}e^{ta} = E^{g(u)}
\]
where $u = t(b-a)$. Write down Tayloy's theorem for $g$ up to order 2 to get bound.

\script{The punchline: the bound is driven by a worst case}
\end{frame}

\begin{frame}[fragile]
\frametitle{Hoeffding's inequality}
\smallCapGreen{Hoeffding's inequality:}
\[
\P\left( |\overline{Z} - \mu| > \epsilon\right) \leq 2e^{-2n\epsilon^2/(b-a)^2}
\]
\smallCapGreen{Proof sketch:} Let $Z$ be zero mean
\begin{align*}
\P\left( \overline{Z} > \epsilon\right)
& 
=
\P\left( e^{t\overline{Z}} > e^{t\epsilon}\right) \\
& \leq e^{-t\epsilon} \E e^{t\overline{Z}} \\
& = e^{-t\epsilon} \prod_{i=1}^n\E e^{tn^{-1}Z_i} \\
& \leq
e^{-t\epsilon}e^{(t/n)^2(b-a)^2/8} \parenthetical{\quad}{\textrm{Now, minimize over $t$ and symmetrize}}
\end{align*}
\end{frame}

\begin{frame}[fragile]
\frametitle{Hoeffding's inequality: Generalizations}
We can let the upper and lower limits change with $i$: $Z_i \in [a_i,b_i]$

\vsp
Also, we can invert this probability statement into a PAC bound:  with probability at least $1-\delta$
\[
|\overline{Z} - \mu| \leq \sqrt{\frac{c}{2n}\log\left(\frac{2}{\delta}\right)}
\]
where $c = n^{-1}\sum_i (b_i - a_i)^2$

\vsp
Compare to Chebyshev, which has growth
\[
|\overline{Z} - \mu| \leq \sqrt{\frac{\sigma^2}{n\delta}}
\]
\end{frame}

\begin{frame}[fragile]
\frametitle{Hoeffding's inequality: Example}
Let $Y_i \in \{0,1\}$, $X_i \in \R^p$ and $g:\R^p \rightarrow \{0,1\}$ be a classifier.
\vsp

Define the training error to be
\[
\hat{R}(g) = \frac{1}{n} \sum_{i=1}^n \mathbf{1}(Y_i \neq g(X_i))
\]
and 
\[
R(g) = \P (Y_i \neq g(X_i))
\]

Then, $\hat{R}(g) - R(g)$ is zero mean and in the interval $[-1,1]$:
\[
\hat{R}(g) - R(g) \leq \sqrt{\frac{1}{n}\log\left(\frac{2}{\delta}\right)}
\]
with high probability (w.h.p)
\end{frame}

\begin{frame}[fragile]
\frametitle{Refined Hoeffding's inequality}
The previous result was restricted to sample means.  

\vsp
This isn't an essential part of the result, however, and is removed in a generalization
known as \alg{McDiarmid's inequality}

\vsp
Suppose that%\footnote{Here, if we write $z$ or $Z$, it will either mean the $n$-vector or a generic $z_i$ or $Z_i$}
\[
\sup_{Z_1,\ldots,Z_n,Z_i'} \left| f(Z_1,\ldots,Z_n) - f(Z_1,\ldots,Z_{i-1},Z_i',Z_{i+1},\ldots,Z_n) \right| \leq c_i
\]

Then
\[
\P\left( |f(Z) - \E f(Z)| > \epsilon\right) \leq 2e^{-2\epsilon^2/\sum_{i=1}^nc_i}
\]

\end{frame}

\begin{frame}[fragile]
\frametitle{McDiarmid's inequality: Example}
Let $f(Z) = \sup_A|\hat \P(A) - \P(A)|$.  

\begin{align*}
|f(Z) - f(Z') |
& = 
\left|\sup_A|\hat \P(A) - \P(A)|   - \sup_A|\hat \P'(A) - \P(A)| \right| \\
& \leq 
\sup_A\left||\hat \P(A) - \P(A)|   - |\hat \P'(A) - \P(A)| \right| \\
& \leq
\sup_A\left| \hat \P(A) - \P(A)- (\hat \P'(A) - \P(A)) \right| \\
& =
\sup_A\left| \hat \P(A) - \hat \P'(A) \right|  \parenthetical{\quad}{| \,|a| - |b|\,|\leq |a - b|}
\end{align*}
\script{Either $Z_i$ is still in $A$ and nothing changes, or $Z_i$ is no longer in $A$ and hence the difference is $1/n$}

\vsp
Therefore
\[
\P\left( |f(Z) - \E f(Z)| > \epsilon\right) \leq 2e^{-2n\epsilon^2}
\]

\end{frame}

\begin{frame}[fragile]
\frametitle{Sharper inequalities}
These previous results don't used any information about \alo{where} the probabilities mass lies

\vsp
Hoeffding's inequality is driven by the worst case: a R.V. that puts all of its mass at the boundaries

\vsp
If the variance of $Z_i$ is small, we can get sharper inequalities 
\end{frame}

\begin{frame}[fragile]
\frametitle{Sharper inequalities}

This idea is that $\sum_{i=1}^n Z_i$ it approximately normally distributed with variance $v = \sum_{i=1}^n \V Z_i$

\vsp
The tails of a $N(0,v)$ are of order $e^{-x^2/(2v)}$

\vsp
\alg{Bernstein's inequality} gives a tail bound that is a combination of a normal and a \alo{penalty} for non-normality
\end{frame}
\begin{frame}[fragile]
\frametitle{Sharper inequalities}

\smallCapGreen{Lemma:}
Suppose that $|X| < c$ and $\E X = 0$. Then for any $t > 0$
\[
\E[e^{tX}] \leq \exp\left\{ t^2 \sigma^2 \left( \frac{e^{tc} - 1 -tc}{(tc)^2} \right) \right\}
\]
where $\sigma^2 = \V X$

\vsp
\smallCapGreen{Idea:} The main part of the proof relies on the inequality: for $r \geq 2$
\[
\E X^r = \E X^{r-2}X^2 \leq c^{r-2}\sigma^2
\]

\script{all higher moments than the variance are killed by the a.s. bound, while the first two moments
are computed as usual}
\end{frame}


\begin{frame}[fragile]
\frametitle{Sharper inequalities}
\smallCapGreen{Bernstein's inequality:}
If $|Z_i| \leq c$ a.s. and $\E Z_i = \mu$, then for all $\epsilon > 0$
\[
\P\left( |\overline{Z} - \mu| > \epsilon\right) \leq 2e^{-\frac{n\epsilon^2}{2\sigma^2 + 2c\epsilon/3}}
\]
where $\sigma^2 = \frac{1}{n} \sum_{i=1}^n\V Z_i$
\vsp

Compare to Hoeffding's:
\[
\P\left( |\overline{Z} - \mu| > \epsilon\right) \leq 2e^{-\frac{n\epsilon^2}{2c^2}}
\]
\end{frame}


\begin{frame}[fragile]
\frametitle{Sharper inequalities}
\begin{itemize}
\item If $\sigma^2 >> 2c\epsilon/3$, then 
\[
\log(\textrm{Bernstein}) \asymp -\frac{n\epsilon^2}{2\sigma^2} \leq -\frac{n\epsilon^2}{4c^2}  \asymp \log(\textrm{Hoeffding})
\]
\item If $\sigma^2 << 2c\epsilon/3$, then 
\[
\log(\textrm{Bernstein}) \asymp -\frac{3n\epsilon^2}{4c\epsilon} =  -\frac{3n\epsilon}{4c} \leq -\frac{n\epsilon^2}{4c^2}  \asymp \log(\textrm{Hoeffding})
\]
\end{itemize}

\smallCapGreen{Note:}  This implies that the Bernstein bound is like an exponential for large $\epsilon$
and normal for small $\epsilon$
\end{frame}

\begin{frame}[fragile]
\frametitle{Bernstein's inequality}
There is a related \alo{moment} version of Bernstein's inequality:

\vsp
All that is really required out of the almost sure boundedness of $Z_i$ is bounds for the moments

\vsp
Suppose that $Z_i$ are such that $\E|Z_i|^m \leq m!M^{m-2}c_i/2$ for all $m \geq 2$ and constants $M, c_i$.  Then
\[
\P\left( |\overline{Z} - \mu| > \epsilon\right) \leq 2e^{-\frac{n\epsilon^2}{2\sigma^2 + 2c\epsilon}}
\]
for $\sigma^2 \geq \frac{1}{n} \sum_{i=1}^n c_i$
\end{frame}

\begin{frame}[fragile]
\frametitle{Bernstein's inequality}
The most useful part of Bernstein's inequality is the associated PAC bound

\vsp
With probability at least $1-\delta$ 
\[
|\overline{Z} - \mu| \leq \sqrt{\frac{2\sigma^2\log(1/\delta)}{n}} + \frac{2c\log(1/\delta)}{3n}
\]
\script{$|Z_i| \leq c$, $\sigma^2 = n^{-1} \sum_{i=1}^n \V Z_i$}

\vsp
In particular, if the variance is small enough:
\[
\sigma^2 \leq \frac{2c^2\log(1/\delta)}{9n}
\quad \Rightarrow \quad |\overline{Z} - \mu| \leq \frac{4c\log(1/\delta)}{3n}
\]
\script{That is, we get a $n$-decay instead of $\sqrt{n}$-decay}
\end{frame}

\transitionSlide{Bounding maximums}

\begin{frame}[fragile]
\frametitle{Finite maximums}
Recall that 
\begin{align*}
\P(|X| > x) 
& \leq
\begin{cases}
\norm{X}_px^{-p} & \textrm{ if } \psi(x) = x^p \\
\frac{1}{e^{(x/\norm{X}_{\psi})^p}-1}  \asymp e^{-(x/\norm{X}_{\psi})^{p}}&  \textrm{ if } \psi(x) = \psi_p(x)
\end{cases}
\end{align*}

Hence, bounds on the $\sup$ of a norm, provide bounds on the probability 
\end{frame}

\begin{frame}[fragile]
\frametitle{Finite maximums}
For the $L_p$ norm, a straight-forward bound exists

\vsp
Suppose that we have the process $X_t$ on $|T| < \infty$

\vsp
Using the fact that $\max |X_t|^p \leq \sum |X_t|^p$, it follows that
\begin{align*}
\norm{\max_{t \in T} X_t}_p 
& =
\left( \E \max_{t\in T} |X_t|^p \right)^{1/p}  \\
& \leq 
\left(  \sum_{t\in T} \E|X_t|^p \right)^{1/p}  \\
 & \leq
|T|^{1/p} \max_{t\in T} \norm{X_t}_p 
\end{align*}
\end{frame}

\begin{frame}[fragile]
\frametitle{Finite maximums}
This continues to hold, but with more complicated proof, for many Orlicz norms

\vsp
Under some technical conditions on $\psi$ that include $\psi_p$ 
\[
\norm{\max_{t \in T} X_t}_\psi \leq K \psi^{-1}(|T|) \max_{t \in T} \norm{X_t}_\psi
\]
Here, $K$ only depends on $\psi$
\end{frame}

\begin{frame}[fragile]
\frametitle{Finite maximums}
Some observations:

\begin{itemize}

\item As $\psi_p^{-1}(|T|) = (\log(1+|T|))^{1/p}$, we get a \alo{logarithmic} increase in $|T|$ while using the $p$-Orlicz norm,
compared to \alo{polynomial} for $L_p$

\item 
This conversion is useless when $|T| = \infty$.  This case can be handled via \alg{generic chaining} whereby each R.V.
is written as a sum of parts of the index space where
\begin{itemize}
\item The R.Vs have low correlation between partitions
\item There aren't too many in each partition
\end{itemize}
\end{itemize}
For a random stochastic process $(X_t)_{t\in T}$, the number of \alo{links} in the chain depends on the (metric) entropy
of $T$

\vsp
This is quantified via \alg{covering}, \alg{packing}, and \alg{bracketing} numbers
\end{frame}

\begin{frame}[fragile]
\frametitle{Covering and packing numbers}
A pseudo-metric space $(T,d)$ is a set with a function $d:T\times T \rightarrow [0,\infty)$ such that
\begin{enumerate}
\item $d(t,t) = 0$
\item $d(s,t) = d(t,s)$
\item $d(s,u) \leq d(s,t) + d(t,u)$
\end{enumerate} 
\script{The pseudo part comes from not insisting that $d(s,t) = 0 \Rightarrow t=s$}
\vsp

\smallCapGreen{Example:} Define the metric related to the empirical measure
\[
d(f,g) = \frac{1}{n}\sum_{i=1}^n |f(Z_i) - g(Z_i)|
\]
Then $d$ is a pseudo-metric, but not a metric
\end{frame}


\begin{frame}[fragile]
\frametitle{Covering numbers}
An $\epsilon$-\alg{cover} is a set $\tilde{T}$ comprised of $\epsilon$-balls such that $T \subseteq \tilde{T}$.
The \alg{covering number} of $T$ is:

\[
N(\epsilon,T,d) = \min\{|\tilde{T}|: \tilde{T} \textrm{ is an } \epsilon-\textrm{cover}\}
\]
Note that 
\begin{itemize}
\item $T$ is \alg{totally bounded} if $N(\epsilon,T,d) < \infty$ for all $\epsilon > 0$
\item The function $\epsilon \mapsto \log N(\epsilon,T,d)$ is the \alg{metric entropy} of $T$
\end{itemize}
\end{frame}

\begin{frame}[fragile]
\frametitle{Packing numbers}
An \alg{$\epsilon$-packing} of $T$ is a subset $\tilde{T}$ comprised of non-overlapping $\epsilon$-balls.
The \alg{packing number} is
\[
M(\epsilon,T,d) = \max\{ |\tilde{T}| : \tilde{T} \textrm{ is an } \epsilon-\textrm{packing of } T\}
\]
\end{frame}

\begin{frame}[fragile]
\frametitle{Covering and packing numbers}
For almost all purposes, the difference between these two concepts are unimportant:

\vsp
For instance: $\forall \epsilon > 0$
\[
M(\epsilon) \leq N(\epsilon)
\]

\vsp
Related lower bounds for $M$ in terms of $N$ are possible with scalar transformations of $\epsilon$
\end{frame}

\begin{frame}[fragile]
\frametitle{Covering and packing numbers}
Computing either covering or packing numbers can be very difficult

\vsp
I advise looking up known results if you go this route, such as
\begin{itemize}
\item Covering numbers of the unit ball of $\R^d$ (such as with ridge or lasso)
\item Compact sets of functions
\end{itemize}
\end{frame}


\transitionSlide{Examples}

\begin{frame}[fragile]
\frametitle{Example: Uniform risk}
Suppose we want to minimize $\hat\P\ell_f = n^{-1} \sum_{i=1}^n |f(X_i) - Y_i|^2$ over a set of functions $\F_n$

\vsp
The main tool in showing consistency/rates of convergence for such estimators is to show
that the \alo{empirical risk} looks like the \alo{true risk}, uniformly over $\F_n$:
\[
\sup_{f \in \F_n} \left| \hat\P\ell_f - \P\ell_f \right| \rightarrow 0 \textrm{ a.s.}
\]
It's easier to generate the set of functions\footnote{Often, $\mathcal{L}_n$ is closed} $\mathcal{L}_n =\{\ell_f : f\in \F_n\}$ and look at
\[
\sup_{\ell \in \mathcal{L}_n} \left| \hat\P\ell - \P\ell \right|
\]
\end{frame}

\begin{frame}[fragile]
\frametitle{Example: Uniform risk }
Using Hoeffding's inequality shows that if $\ell(Z) \in [0,b]$ a.s., then
\[
\P\left( \left|  \hat\P\ell - \P\ell \right| > \epsilon \right) \leq 2 \exp\left\{ -\frac{2n\epsilon^2}{b^2}\right\}
\]
By a union bound (with $\mathcal{L}$ finite)
\[
\P\left(\sup_{\ell \in \mathcal{L}_n} \left|  \hat\P\ell - \P\ell \right| > \epsilon \right) \leq 2 |\mathcal{L}_n| \exp\left\{ -\frac{2n\epsilon^2}{b^2}\right\}
\]
\script{$\P \cup_{j=1}^J A_j \leq \sum_{j=1}^J \P(A_j)$}

\vsp
This pairing is quite common in basic theory
\end{frame}

\begin{frame}[fragile]
\frametitle{Example: Uniform risk }
This gives us an \alo{in probability} convergence

\vsp
This can be strengthened to an \alo{almost sure} convergence via the \alg{Borel-Cantelli} lemma

\vsp
If $|\mathcal{L}_n|$ grows slowly enough for
\[
\sum_{n=1}^\infty |\mathcal{L}_n| \exp\left\{ -\frac{2n\epsilon^2}{b^2}\right\} < \infty
\]
Then
\[
\sup_{\ell \in \mathcal{L}_n} \left| \hat\P\ell - \P\ell \right| \rightarrow 0 \textrm{ a.s.}
\]
\end{frame}

\begin{frame}[fragile]
\frametitle{Example: Uniform risk }
Though this will work for some situations, $|\mathcal{L}_n|$ is really infinite

\vsp
In this case, the goal becomes to find a finite set $\mathcal{L}_{n,\epsilon}$ such that
\begin{align*}
\left\{ \sup_{\ell \in \mathcal{L}_n} \left| \hat\P\ell - \P\ell \right| > \epsilon \right\} & \subseteq
\left\{ \sup_{\ell \in \mathcal{L}_{n,\epsilon}} \left| \hat\P\ell - \P\ell \right| > \epsilon' \right\}
\end{align*}
Letting $\epsilon'$ be a function of $\epsilon$ but not $n$

\vsp
We construct $\mathcal{L}_{n,\epsilon}$ using covering numbers with respect to different metrics $\norm{\cdot}$

\script{This corresponds to $(T,d) \leftrightarrow (\mathcal{L}_{n},\norm{\cdot})$}
\end{frame}

\begin{frame}[fragile]
\frametitle{Example: Uniform risk }
The most  basic choice is 
\[
d_{\infty}(\ell,\ell') = \norm{\ell - \ell'}_{\infty} = \sup_z| \ell(z) - \ell'(z)|
\]
An $\epsilon$-cover of $\mathcal{L}_n$ with respect to $d_{\infty}$ is such that for every $\ell \in \mathcal{L}_n$,
$\exists \ell_{\epsilon} \in \mathcal{L}_{n,\epsilon}$ such that
\[
d_{\infty}(\ell,\ell_\epsilon) = \norm{\ell - \ell_\epsilon}_{\infty} = \sup_z| \ell(z) - \ell_\epsilon(z)| < \epsilon
\]

\vsp
We'll define the $L_{\infty}$ \alo{covering number} 

\[
N_\infty(\epsilon,\mathcal{L}_n) = N_\infty(\epsilon,\mathcal{L}_n,\norm{\cdot}_\infty)
\]
to be the minimal $\epsilon$-cover

\script{Alternatively known as the uniform covering number (not to be confused with uniform convergence)}

\end{frame}

\begin{frame}[fragile]
\frametitle{Example: Uniform risk }
Putting this together, we get
\[
\P\left(\sup_{\ell \in \mathcal{L}_n} \left|  \hat\P\ell - \P\ell \right| > \epsilon \right) \leq 2 N_\infty(\epsilon/3,\mathcal{L}_n) \exp\left\{ -\frac{2n\epsilon^2}{9b^2}\right\}
\]
\smallCapGreen{Basic idea:} Let $\mathcal{L}_{n,\epsilon/3}$ be a minimal $\epsilon/3$-cover of $\mathcal{L}_n$.
Fix a $\ell \in \mathcal{L}_n$, then $\exists \ell' \in \mathcal{L}_{n,\epsilon/3}$ such that $\norm{\ell - \ell'}_\infty < \epsilon/3$
\begin{align*}
\left|  \hat\P\ell - \P\ell \right|  
& \leq
\left|  \hat\P\ell - \hat\P\ell' \right|  + \left|  \hat\P\ell' - \P\ell' \right|   + \left|  \P\ell' - \P\ell \right|  \\
& \leq
\norm{\ell - \ell'}_{\infty} + \left|  \hat\P\ell' - \P\ell' \right|   + \norm{\ell - \ell'}_{\infty} \\
& \leq
2\epsilon/3 + \left|  \hat\P\ell' - \P\ell' \right|   
\end{align*}
Now, we do as before, but with $N_\infty(\epsilon/3,\mathcal{L}_n) = |\mathcal{L}_{n,\epsilon/3}|$ 
\end{frame}

\begin{frame}[fragile]
\frametitle{Example: Uniform risk }
Unfortunately, $N_\infty(\epsilon/3,\mathcal{L}_n)$ is often too large for
\[
\sum_{n=1}^\infty N_\infty(\epsilon/3,\mathcal{L}_n) \exp\left\{ -\frac{2n\epsilon^2}{9b^2}\right\} < \infty
\]
\script{The uniform norm is quite strong}

\vsp
At issue is that
\[
\left|  \hat\P\ell - \hat\P\ell' \right|    + \left|  \P\ell' - \P\ell \right|  
 \leq
\norm{\ell - \ell'}_{\infty}  + \norm{\ell - \ell'}_{\infty} 
\]
is quite coarse

\vsp
A solution is to appeal to \alo{random $L_1$ norm covers}
\end{frame}

\begin{frame}[fragile]
\frametitle{Example: Uniform risk }
The idea is based around fictively creating a \alg{ghost sample} $Z_{1},\ldots,Z_{2n}$  such that
$Z_i \stackrel{i.i.d}{\sim} \P$ for $i=1,\ldots,2n$

\vsp
Now, we will think of $\P \ell \approx \hat\P_{n+1}^{2n} \ell$ and form
\[
\left\{ \sup_{\ell \in \mathcal{L}_n} \left| \hat\P\ell - \ \hat\P_{n+1}^{2n}\ell\right| > \epsilon \right\}  \subseteq
\left\{ \sup_{\ell \in \mathcal{L}_{n,\epsilon}} \left| \hat\P\ell -  \hat\P_{n+1}^{2n}\ell \right| > \epsilon' \right\}
\]
where now $\mathcal{L}_{n,\epsilon}$ is going to be a data-dependent set 

\script{i.e. a random variable}

\vsp
Now, we cover $\mathcal{L}_n$ with $\epsilon$-covers with respect to
\[
\norm{\ell - \ell'}_{n} = \int |\ell - \ell'| d\hat\P_{1}^{n}
\]
with covering number $N_1(\epsilon,\mathcal{L}_n) = N(\epsilon,\mathcal{L}_n,\norm{\cdot}_{n})$
\end{frame}

\begin{frame}[fragile]
\frametitle{Example: Uniform risk }
The randomness of the covering number can be dealt with by following four steps

\begin{enumerate}
\item \smallCapGreen{Ghost sample}
\item \smallCapGreen{Introduction of additional randomness} via Rademacher random variables
\item \smallCapGreen{Conditioning} to introduce covering number
\item \smallCapGreen{Hoeffding}
\end{enumerate}
Let's go over each of these briefly
\end{frame}

\begin{frame}[fragile]
\frametitle{Example: Uniform risk }
\smallCapGreen{Ghost sample:} It can be shown that for $n \geq 2b^2/\epsilon^2$
\[
\P\left(\sup_{\ell \in \mathcal{L}_n} \left|  \hat\P\ell - \P\ell \right| > \epsilon \right) 
\leq 
2\P\left(\sup_{\ell \in \mathcal{L}_n} \left|  \hat\P\ell - \hat\P_{n+1}^{2n}\ell \right| > \epsilon/2 \right) 
\]
\script{See pages 136-138 of Gy\"orfi et al. (2002) for details}
\end{frame}

\begin{frame}[fragile]
\frametitle{Example: Uniform risk }
\smallCapGreen{Introduction of additional randomness:} 
As all the R.V.s are iid, their difference is invariant to a random sign change with Rademacher R.V.s
\begin{align*}
\lefteqn{\P\left(\sup_{\ell \in \mathcal{L}_n} \left|  \hat\P\ell - \hat\P_{n+1}^{2n}\ell \right| > \frac{\epsilon}{2}\right) } \\
& =
\P\left(\sup_{\ell \in \mathcal{L}_n} \left| \sum_{i=1}^n \epsilon_i \left(\ell(Z_i) - \ell(Z_{i + n})\right) \right| > \frac{n\epsilon}{2} \right)  \\
& \leq
\P\left(\sup_{\ell \in \mathcal{L}_n} \left| \sum_{i=1}^n \epsilon_i \ell(Z_i)\right| > \frac{n\epsilon}{4}\right)
+
\P\left(\sup_{\ell \in \mathcal{L}_n} \left| \sum_{i=1}^n \epsilon_i \ell(Z_{i+n})\right| > \frac{n\epsilon}{4}\right)\\
& =
2\P\left(\sup_{\ell \in \mathcal{L}_n} \left| \sum_{i=1}^n \epsilon_i \ell(Z_i)\right| > \frac{n\epsilon}{4}\right)
\end{align*}
\script{This is known as \alg{symmetrization}.  We'll return to this again}
\end{frame}

\begin{frame}[fragile]
\frametitle{Example: Uniform risk }
\smallCapGreen{Conditioning:} To introduce the covering number, observe that
\begin{align*}
\lefteqn{\P\left(\sup_{\ell \in \mathcal{L}_n} \left| \sum_{i=1}^n \epsilon_i \ell(Z_i)\right| > \frac{n\epsilon}{4}\right)} \\
& =
\E_Z\P_\epsilon\left(\exists \ell \in \mathcal{L}_n : \left| \sum_{i=1}^n \epsilon_i \ell(z_i)\right| > \frac{n\epsilon}{4} \bigg| (Z_i)_{i=1}^n = (z_i)_{i=1}^n\right)
\end{align*}
Now, we can apply the complexity part: Let $\mathcal{L}_{n,\epsilon/8}$  be an $\epsilon/8$-cover of $\mathcal{L}_n$
with respect to $\norm{\cdot}_{n}$:
\[
\frac{1}{n}\sum_{i=1}^n |\ell(z_i) - \ell'(z_i)| < \epsilon/8
\]
\end{frame}

\begin{frame}[fragile]
\frametitle{Example: Uniform risk }
\smallCapGreen{Conditioning:} Fix an $\ell$ and find its $\ell' \in \mathcal{L}_{n,\epsilon/8}$,
\begin{align*}
\left| \frac{1}{n}\sum_{i=1}^n \epsilon_i \ell(z_i)\right| 
& \leq
\left| \frac{1}{n}\sum_{i=1}^n \epsilon_i \ell'(z_i)\right| + \epsilon/8
\end{align*}
Showing (conditional on $(Z_i)_{i=1}^n = (z_i)_{i=1}^n$)
\begin{align*}
\lefteqn{\P\left(\exists \ell \in \mathcal{L}_{n,\epsilon/8} : \left| \frac{1}{n}\sum_{i=1}^n \epsilon_i \ell(z_i)\right| + \epsilon/8 > \frac{\epsilon}{4} \right)} \\
& \leq 
N_1(\epsilon,\mathcal{L}_n) \max_{\ell \in \mathcal{L}_{n,\epsilon/8}}
\P\left( \left| \frac{1}{n}\sum_{i=1}^n \epsilon_i \ell(z_i)\right| >\frac{\epsilon}{8}\right)
\end{align*}
\end{frame}

\begin{frame}[fragile]
\frametitle{Example: Uniform risk }
\smallCapGreen{Hoeffding:} Lastly, we just need to bound 
\[
\P\left( \left| \frac{1}{n}\sum_{i=1}^n \epsilon_i \ell(z_i)\right| >\frac{\epsilon}{8}\right)
\]
This can be done with Hoeffding

\script{Recall, though, that Bernstein's tends to give better bounds}
\[
\P\left( \left| \frac{1}{n}\sum_{i=1}^n \epsilon_i \ell(z_i)\right| >\frac{\epsilon}{8}\right)
\leq
2\exp\left\{- \frac{2n(\epsilon/8)^2}{(2b)^2} \right\}
\leq
2\exp\left\{- \frac{2n\epsilon^2}{128b^2} \right\}
\]
\end{frame}

\begin{frame}[fragile]
\frametitle{Example: Uniform risk }
\smallCapGreen{In conclusion:}
\[
\P\left(\sup_{\ell \in \mathcal{L}_n} \left|  \hat\P\ell - \P\ell \right| > \epsilon \right) 
\leq 
8\E N_1(\epsilon,\mathcal{L}_n) \exp\left\{- \frac{2n\epsilon^2}{128b^2} \right\}
\]

\vsp
Now, we need to know the expected $L_1$ covering number

\vsp
This can be difficult to calculate.  Hence, we can turn to a few notions
\begin{itemize}
\item VC dimension
\item Bracketing numbers
\end{itemize}
\end{frame}

\begin{frame}[fragile]
\frametitle{Example: Uniform risk }
Recall the VC dimension of a class of sets $\mathcal{A}$
is the largest number of points $\mathcal{G}_n$ which we can shatter

\script{That is, the largest $n$ such that  $|\{G \subset \mathcal{G}_n : G = \mathcal{G}_n \cap A, A \in \mathcal{A}\}| = 2^n$} 
\vsp

Let 
\[
\mathcal{G} = \{ (z,t) \in \R^{p+1}\times\R: t \leq \ell(z), \ell \in \mathcal{L}_n\}
\]
\script{This is the set of all \alg{subgraphs} of functions in $\mathcal{L}_n$}

\vsp
We can bound the $L_q$ packing numbers by $VC_{\mathcal{G}} =$  VC dimension of $\mathcal{G}$
\end{frame}

\begin{frame}[fragile]
\frametitle{Example: Uniform risk }
Let $\nu$ be a probability measure on $\R^{p+1}$ and let $\sup_{\ell \in \mathcal{L}_n} \norm{\ell}_{\infty} \leq b$.  Then
for any $\epsilon \in (0,b/4)$
\[
M(\epsilon,\mathcal{L}_n,\norm{\cdot}_{L_q(\nu)}) \leq 
3
\left(
\frac{2eb^q}{\epsilon^q}\log\left( \frac{3eb^q}{\epsilon^q}\right) 
\right)^{VC_{\mathcal{G}}}
\]
\script{Remember: packing numbers are essentially covering numbers for our purposes. Note that the bound on
the RHS doesn't depend on $\nu$}

\vsp
Hence, we can leverage the ``agreement'' between the empirical $L_q$ norm and packing numbers to bound 
the random quantity with the nonrandom VC dimension

\vsp
Unfortunately, 
\begin{itemize}
\item the gap between these bounds can be huge
\item VC dimension can frequently only be upper-bounded itself
\end{itemize}
\end{frame}

\begin{frame}[fragile]
\frametitle{Example: Uniform risk }
Alternatively, we can use  \alg{bracketing numbers}

\vsp
For a given function class $\F$, we can make a \alg{bracket}:
\[
[L,U] = \{f \in \F: L(x) \leq f(x) \leq U(x), \forall x\}
\]
and a \alg{bracketing} of $\F$ is a collection of brackets such that
\[
\F \subseteq \bigcup [L_j,U_j]
\]

We are most interested in $\epsilon-L_q(\P)$-bracketings: for all $j$
\[
\left(\int |U_j - L_j|^q d\P\right)^{1/q} \leq \epsilon
\]
This smallest $\epsilon-L_q(\P)$-bracketing is the \alg{bracketing number}
\[
N_{[\,]}(\epsilon, \F, \norm{\cdot}_{L_q(\P)})
\]
\end{frame}

\begin{frame}[fragile]
\frametitle{Example: Uniform risk }
Bracketing numbers are a bit larger that covering numbers
\[
N(\epsilon, \F, \norm{\cdot}_{L_q(\P)}) \leq N_{[\,]}(2\epsilon, \F, \norm{\cdot}_{L_q(\P)})
\]
But they provide stronger control over complexity.  To wit:

\vsp
\smallCapGreen{Fundamental GC theorem:}  If $N_{[\,]}(\epsilon, \F, \norm{\cdot}_{L_1(\P)}) < \infty$ for all $\epsilon > 0$.  Then
$\F$ is GC

\vsp
\smallCapGreen{Proof idea:} For any $f \in \F$, $\exists j$
\[
(\hat\P - \P)f \leq (\hat\P - \P)U_j + (\P U_j - f) \leq (\hat\P - \P)U_j + \epsilon
\]
Hence
\[
\sup_{f \in \F} (\hat\P - \P)f  \leq \max_j (\hat\P - \P)U_j + \epsilon \leq 2\epsilon \parenthetical{\quad}{\textrm{a.s. for large enough $n$}}
\]

\end{frame}

\begin{frame}[fragile]
\frametitle{Example: Uniform risk }
We can get an exponential bound using bracket, just like for covering numbers/VC dimension:

\[
\P\left( \sup_{\ell \in \mathcal{L}_n}|(\hat\P - \P)\ell| > \epsilon\right) 
\leq
4N_{[\,]}(\epsilon, \F, \norm{\cdot}_{L_1(\P)})\exp\left\{ -\frac{96n\epsilon^2}{76Fb}\right\}
\]
where 
\begin{itemize}
\item $F = \sup_\ell \norm{\ell}_{L_1(\P)}$
\item $b = \sup_{\ell} \norm{\ell}_{L_\infty(\P)}$
\end{itemize}

\vsp
The main utility of this bound is that, in my experience, bracketing numbers are easier to bound/compute
\end{frame}

\begin{frame}[fragile]
\frametitle{Example: Uniform risk }
\smallCapGreen{Important case: Lipschitz in a parameter} 

\vsp
Suppose the $\mathcal{L} = \{ \ell_\beta: \beta \in \mathcal{B}\}$. 
\vsp

For example, for regression:
\[
\ell_\beta(Z) = (Y - X^{\top}\beta)^2
\]

\vsp
So, the squared error loss class is indexed by the regression coefficients

\vsp
Sometimes, the complexity of $\mathcal{L}$ can be translated to the complexity of $\mathcal{B}$

\script{For constrained least squares, this is a huge win as the complexity of norm balls in Euclidean space is well known}
\end{frame}

\begin{frame}[fragile]
\frametitle{Example: Uniform risk }
\smallCapGreen{Important case: Lipschitz in a parameter} 
Intuitively, a bracket on $\mathcal{B}$ can make a bracket on $\mathcal{L}$, provided $\ell_\beta - \ell_{\beta'}$
can't change too much relative to $\beta - \beta'$.  

\vsp
Translated into mathematics, this is a Lipschitz-type constraint:  

\vsp
Let $(\mathcal{B},\norm{\cdot})$ be a normed subset of
$\R^{p+1}$.  If there is a function $m$ where
\[
|\ell_{\beta}(z) - \ell_{\beta'}(z)| \leq m(z)\norm{\beta - \beta'}
\]
then
\[
N_{[\,]}(\epsilon, \F, \norm{\cdot}_{L_q(\P)}) \leq \left( \frac{4\sqrt{p+1} \; \textrm{diam}(\mathcal{B}) \norm{m}_{L_q(\P)}^q}{\epsilon} \right)^{p+1}
\]
\script{Often this approach won't work in ``high dimensions'' as the dimension exponential dominates the sample
size exponential}
\end{frame}

\begin{frame}[fragile]
\frametitle{Example: Uniform risk }
To get a more careful bound, we need the \alg{contraction theorem}

\script{See Ledoux and Talagrand (1991). Often we need to make more assumptions to bound the symmetrized process,
as in the last example in this lecture}

\vsp
If the loss $\ell$ is Lipschitz, then for any function $f_* \in \F$ and nonrandom $z_i$
\begin{align*}
\lefteqn{\E\left( \sup_{f \in \F} \left| \sum_{i=1}^n \epsilon_i\left( \ell_f(z_i) - \ell_{f_*(z_i)} \right) \right| \right) }\\
& \leq
2\E\left( \sup_{f \in \F} \left| \sum_{i=1}^n \epsilon_i\left( f(z_i) -f_*(z_i) \right) \right| \right) 
\end{align*}
\script{See van der Geer (2008) for an interesting application of this technique}
\end{frame}

%
%\begin{frame}[fragile]
%\frametitle{Example: Improving Greenshtein}
%Suppose we make a condition on $\ell_\beta(Z) = (Y - X^{\top}\beta)^2$ to make it Lipschitz. 
%
%\vsp
% This usually involves some combination of assuming
%\begin{itemize}
%\item $Y$ has finite norm 
%\item $X$  has finite norm
%\item $\sup_{\ell \in \mathcal{L}} \norm{\ell}_{\infty} < L < \infty$
%\end{itemize}
%As
%\[
%\nabla_\beta \ell_\beta(Z) = -2YX + 2 XX^{\top}\beta = 2X(X^{\top}\beta - Y)
%\]
%and
%\[
%|\ell_\beta(Z) - \ell_{\beta'}(Z)| \leq \sup_{\beta \in \mathcal{B}}\norm{  \nabla_{\beta}\ell_\beta(Z)}_r \norm{\beta - \beta'}_s
%\]
%where $r$ and $s$ are H\"older constants
%\end{frame}
%\begin{frame}[fragile]
%\frametitle{Example: Improving Greenshtein}
%If we're looking at lasso, then we choose $s = \infty$ and $r = 1$
%
%\vsp
%Fix a $t$, then $\textrm{diam}(\mathcal{B},\norm{\cdot}_1) = t$
%\vsp
%
%Make norm assumptions to make the $\sup_\beta$ Lipschitz constant finite
%
%\vsp 
%
%Then
%\[
%\left( \frac{4\sqrt{p+1} \; \textrm{diam}(\mathcal{B}) \norm{m}_{L_q(\P)}^q}{\epsilon} \right)^{p+1} \lesssim
%\left( \frac{C t\sqrt{p+1} }{\epsilon} \right)^{p+1}
%\]
%
%\end{frame}
\transitionSlide{Sub-Gaussian bounds}

\begin{frame}[fragile]
\frametitle{Sub-Gaussian}
A \alg{sub-Gaussian} random variable is one that has tail decay at least as fast a Gaussian

\vsp
Note that $X\sim N(0,1)$ has $\P(|X| > x) \leq 2 \exp\{-x^2/2\}$ and in fact $\norm{X}_{\psi_2} = \sqrt{8/3}$

\vsp
Any R.V. that obeys this tail quantity is a sub-Gaussian R.V.

\vsp
A sub-Gaussian \alo{process} is a $\{X_t\}_{t\in T}$ where
\[
\P(|X_t - X_s| > x) \leq 2\exp\left\{ -\frac{x^2}{2d^2(s,t)}\right\}
\]
Note that a process is sub-Gaussian with respect to the pseudo-metric on the index set

\script{This is crucial to remember}
\end{frame}

\begin{frame}[fragile]
\frametitle{Sub-Gaussian}
\smallCapGreen{Reminder:} $\P(|X| > x) \leq C e^{-cx^p}$ iff $\norm{X}_{\psi_p} < \infty$.  Hence:
\[
\norm{X_s -X_t}_{\psi_2} \leq \sqrt{6} d(s,t)
\]
\script{Here, we can see our approach.  If our process isn't necessarily Lipschitz, but it is sub-Gaussian, then
it is Lipschitz with respect to the $\psi_2$-Orlicz norm and $d$}

\vsp
\smallCapGreen{Example:} Any Gaussian process is sub-Gaussian with respect to the (standard deviation) 
pseudo-metric $d(s,t) = \sigma(X_s - X_t)$ 

\vsp
\smallCapGreen{Example:} Let $\epsilon_1,\ldots,\epsilon_n$ be i.i.d uniform($\{-1,1\}$) and $a = (a_1,\ldots,a_n)^{\top} \in \R^n$.  Then
\[
X_a = \sum_{i=1}^n a_i \epsilon_i
\]
is a sub-Gaussian process w.r.t. $d(a,b) = \norm{a-b}_2$ known as the \alg{Rademacher process}
\end{frame}

\begin{frame}[fragile]
\frametitle{Sub-Gaussian}
This last example follows from an important special case of Hoeffding's inequality
\[
\P(|X_a| > x) \leq 2 \exp\left\{ -\frac{x^2}{2\norm{a}_2^2}\right\}
\]

\vsp
It turns out the innocuous looking Rademacher process is \alo{crucial}.  But first, let's state the
main result
\end{frame}



\begin{frame}[fragile]
\frametitle{Sub-Gaussian}
Suppose we sub-Gaussian process $(X_t)_{t\in T}$ such that $\norm{X_s -X_t}_{\psi_2} \leq C d(s,t)$

\vsp
Also, let the diameter of $T$ be 
\[
D = \sup_{s,t \in T} d(s,t)
\]
Then\footnote{There are some other technical conditions, notably separability,
to this result (the $\sup$ of a measurable process
isn't necessarily measurable)} if $X_t$ is zero mean
\[
\E \sup_{t \in T} X_t \leq K\int_0^D \sqrt{\log(N(\epsilon,T,d))} d\epsilon
\]
\script{See Chapter 1.2 of Talagrand's Generic Chaining. For the symmetrizing statement for lower bounds, see
Lemma 1.2.8}
\end{frame}

\begin{frame}[fragile]
\frametitle{Sub-Gaussian uniform bound}
\[
\E \sup_{t \in T} X_t \leq K\int_0^D \sqrt{\log(N(\epsilon,T,d))} d\epsilon
\]

\smallCapGreen{Note:} The $\sqrt{\log}$ part comes from the inverse of the $\psi_2$ function.  The $\int$ comes
from `adding' together subsets of $T$ 

\vsp
\smallCapGreen{Some notes:}
\begin{itemize}
\item This is known as Dudley's inequality
\item The upper bound can be improved with Talagrand's $\Gamma$ function, as demonstrated in his book Generic Chaining
\item Often, Dudley's bound gives the appropriate rate, but with unnecessary log factors
\end{itemize}

\end{frame}
%
%\begin{frame}[fragile]
%\frametitle{Sub-Gaussian}
%Unfortunately, this result isn't immediately applicable due to many things (in particular the empirical process) not
%being sub-Gaussian
%
%\vsp
%We need a technique to pass to a sub-Gaussian process via \alg{symmetrization}
%
%\end{frame}




\begin{frame}[fragile]
\frametitle{Sub-Gaussian uniform bound}
The most important process is the \alo{empirical process}:

\[
\norm{\hat\P - \P}_\F = \sup_{f \in \F} |\hat\P f - \P f|
\]

\vsp
Unfortunately, the empirical process is not sub-Gaussian and we can't use the previous bound

\vsp
In fact, by Bernstein's inequality, the typical tail behavior is a mixture of sub-Gaussian and sub-exponential
tails

\vsp
The main tool in this case is called \alg{symmetrization}

\script{Theorem 1.2.7 in Generic Chaining provides the relevant upper bound for processes that are mixtures
of $\psi_2$ and $\psi_1$ Orlicz norm bounds.  I've never seen anyone go this route in statistics, though}
\end{frame}

\begin{frame}[fragile]
\frametitle{Sub-Gaussian uniform bound}
For Rademacher R.V.s $\epsilon_i$
\[
\E \norm{\hat\P - \P}_\F \leq \frac{2}{n} \E_Z\E_\epsilon \sup_{f\in \F} \left| \sum_{i=1}^n \epsilon_i f(Z_i) \right|
\]

\vsp
This generates a random process that is conditionally sub-Gaussian (Hoeffding's inequality) on the observed data, indexed by the (random) coordinate evaluation $f(Z_i)$

\vsp
Sub-Gaussian \alo{must} always be stated with respect to a metric on the indexing set.  

\vsp
This turns out to be rather complicated in this case 
\end{frame}

\begin{frame}[fragile]
\frametitle{Sub-Gaussian uniform bound}
Let's return to the generalization error bound for the lasso

\vsp
Previously, we used \alo{Nemirovski's inequality} to bound the process

\vsp
It was conjectured in Greenshtein, Ritov (2004) that the induced $n^{1/4}$ rate could be increased to $n^{1/2}$

\script{Additionally, they showed that under Gaussian design, they could in fact get the $1/2$ rate}

\vsp
Recently, this was answered in the affirmative by Bartlett et al. (2012), using the techniques we have discussed so far

\script{In fact, they were able to make weaker assumptions as well}
\end{frame}

\begin{frame}[fragile]
\frametitle{Sub-Gaussian uniform bound}
The approach is to
\begin{enumerate}
\item Symmetrize with Rademacher r.v.s
\item Bound with Dudley's inequality $\E_\epsilon \sup_{f\in \F} \left| \sum_{i=1}^n \epsilon_i f(Z_i) \right|$
\item Compute the $\sqrt{\log(N,\epsilon,T,d)}$, where $d$ is now the complicated norm from the symmetrization
\item Integrate this bound up to the diameter of the indexing set
\end{enumerate}

\vsp
If we do this, we can increase $t$ like
\[
t_n = o\left( \frac{\sqrt{n}}{\log^{3/2}(n) \log^{3/2}(n p)}\right)
\]
which, compared with Greenstein's result provides a faster rate
\[
t_n = o\left( \left(\frac{n}{\log( p)}\right)^{1/4}\right)
\]
\end{frame}

\begin{frame}[fragile]
\frametitle{Bookkeeping}
\begin{itemize}
\item The best known concentration inequality of the Bernstein-type can be found in Bousquet (2001)
\item The best possible bounds for the $\E \sup_{t \in T} X_t$ is given by Talagrand's $\Gamma$-function,
though I've never seen this used
\item symmetrization $+$ Dudley's bound is the more popular route
\end{itemize}

\end{frame}
%
%\begin{frame}[fragile]
%\frametitle{Next time}
%\begin{figure}
%\centering
%\includegraphics[width=4in]{../figures/neuron.pdf}
%\end{figure}
%\end{frame}
\end{document}
