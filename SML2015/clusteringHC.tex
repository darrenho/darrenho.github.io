\documentclass{beamer}
%\usepackage[usenames,dvipsnames]{xcolor}

\usepackage{_defsAndPackages675notation}
\usepackage{_defsAndPackages675beamer}

\begin{document}

\title{\alg{Clustering}}
\subtitle{\classTitle}
%\author{\alg{Darren Homrighausen, PhD}}
%\institute{\classTitle}
\date{}



\begin{frame}
\maketitle
%\titlepage
%\begin{figure}[h!]
%  \centering
%  \includegraphics[width=1in]{.../../figures/CSU_logo2.eps}
%\end{figure}
%
\organization
%
\end{frame}

\begin{frame}[fragile]
\frametitle{K-means}
\begin{enumerate}
\item Select a number of clusters $K$.
\item Let $C_1,\ldots,C_K$ partition $\{1,2,3,\ldots,n\}$ such that
	\begin{itemize}
		\item All observations belong to some set $C_j$.
		\item No observation belongs to more than one set.
	\end{itemize}
\item K-means attempts to form these sets by making \alo{within-cluster variation}, $W(C_k)$, as small as possible.
\[
\min_{C_1,\ldots,C_K} \sum_{k=1}^K W(C_k).
\]
\item To Define $W$, we need a concept of distance.  By far the most common is Euclidean
\[
W(C_k) =  \frac{1}{|C_k|} \sum_{i,i' \in C_k} ||X_i - X_{i'}||_2^2.
\]
That is, the average (Euclidean) distance between all cluster members.
\end{enumerate}
\end{frame}

\begin{frame}[fragile]
\frametitle{K-means}
It turns out
\begin{equation}
\min_{C_1,\ldots,C_K} \sum_{k=1}^K W(C_k).
\end{equation}
is too hard of a problem to solve computationally ($K^n$ partitions!).

\vsp
So, we make a greedy approximation:
\begin{enumerate}
\item Randomly assign observations to the $K$ clusters
\item Iterate until the cluster assignments stop changing:
\begin{itemize}
\item For each of $K$ clusters, compute the \alg{centroid}, which is the $p$-length vector of the means in that
cluster.
\item Assign each observation to the cluster whose centroid is closest (in Euclidean distance).
\end{itemize}
\end{enumerate}
This procedure is guaranteed to decrease (1) at each step. 
\vsp

\end{frame}


\begin{frame}[fragile]
\frametitle{K-means: A Summary}
To fit K-means, you need to 
\begin{enumerate}
\item Pick $K$ (inherent in the method)
\item Convince yourself you have found a good solution (due to the randomized approach to the algorithm).
\end{enumerate}
\vsp

It turns out that \alb{1.} is difficult to do in a principled way.  We will discuss this next
\vsp

For \alb{2.}, a commonly used approach is to run K-means many times with different starting points.  Pick the
solution that has the smallest value for 
\[
\min_{C_1,\ldots,C_K} \sum_{k=1}^K W(C_k)
\]
\end{frame}

\transitionSlide{Choosing $K$}

\begin{frame}[fragile]
\frametitle{Choosing the Number of Clusters}
Why is it important?
\begin{itemize}
\item It might make a big difference (concluding there are $K = 2$ cancer sub-types versus
$K = 3$).
\item One of the major goals of statistical learning is automatic inference.  A good way of
choosing $K$ is certainly a part of this.
\end{itemize}
\end{frame}

\begin{frame}[fragile]
\frametitle{Reminder: What does $K$-means do?}
Given a number of clusters $K$, we (approximately) minimize:
\[
\sum_{k=1}^K W(C_k) 
%= \sum_{k=1}^K \frac{1}{|C_k|} \sum_{i,i' \in C_k} \sum_{j=1}^p (x_{ij} - x_{i'j})^2 
= \sum_{k=1}^K\frac{1}{|C_k|} \sum_{i,i' \in C_k} ||X_i - X_{i'}||_2^2.
\]
We can rewrite this in terms of the \alo{centroids} as
\[
W(K) = \sum_{k=1}^K  \sum_{i \in C_k}  ||X_i - \overline{X}_k||_2^2,
\]
\end{frame}

\begin{frame}[fragile]
\frametitle{Minimizing $W$ in $K$}
Of course, a lower value of $W$ is better.  Why not minimize $W$?
\begin{blockcode}
plotW = rep(0,49)
for(K in 1:49){
  plotW[K] = 	kmeans(x,centers=K,nstart=20)$tot.withinss
}
\end{blockcode}

\begin{figure}[h!]
  \centering
  \includegraphics[width=2.5in,trim=0 0 0 50,clip]{../figures/clusteringKmeansMinimizeW.pdf}
\end{figure}

\end{frame}

\begin{frame}[fragile]
\frametitle{Minimizing $W$ in $K$}
Of course, a lower value of $W$ is better.  Why not minimize $W$?

\vsp
A look at the cluster solution

\begin{figure}[h!]
  \centering
  \includegraphics[width=2.8in,trim=0 0 0 0,clip]{../figures/clusteringKmeansMinimizeW2.pdf}
\end{figure}
\end{frame}

\begin{frame}[fragile]
\frametitle{Between-cluster variation}
Within-cluster variation measures how \alo{tightly grouped} the clusters are.  As 
we increase $K$, this will always decrease.  
\vsp

What we are missing is \alo{between-cluster variation}, ie: how spread apart the groups are
\[
B = \sum_{k=1}^K |C_k| ||\overline{X}_k - \overline{X} ||_2^2,
\]
where $|C_k|$ is the number of points in $C_k$, and $\overline{X}$ is the grand mean
of all observations:
\[
\overline{X} = \frac{1}{n} \sum_{i=1}^n X_i.
\]
\end{frame}

\begin{frame}[fragile]
\frametitle{Between-cluster variation: Example}
\begin{figure}[h!]
  \centering
\includegraphics[width=2.5in,trim=0 0 0 0,clip]{../figures/clusteringKmeansBetweenSSexample.pdf}
\end{figure}
\begin{itemize}
\item[] B = \textcolor{green!65!black}{$|C_1| ||\overline{X}_1 - \overline{X} ||_2^2$} +
\textcolor{red}{$|C_2| ||\overline{X}_2 - \overline{X} ||_2^2$}
\item[] W = \textcolor{green!65!black}{$\sum_{i \in C_1} |C_1| ||\overline{X}_1 - X_i ||_2^2$} +
\textcolor{red}{$\sum_{i \in C_2} |C_2| ||\overline{X}_2 - X_i ||_2^2$}
\end{itemize}
\end{frame}


%\begin{frame}[fragile]
%\frametitle{R tip detour}
%\begin{figure}[h!]
%  \centering
%\includegraphics[width=1.1in,trim=0 40 0 55,clip]{../figures/clusteringKmeansBetweenSSexample.pdf}
%\end{figure}
%\begin{blockcode}
%x1Bar = apply(x[1:25,],2,mean)
%x2Bar = apply(x[26:50,],2,mean)
%xBar  = apply(x,2,mean)
%
%plot(x,xlab='x1',ylab='x2',col=kmeans(x,centers=2,nstart=20)$cluster+1)
%points(x1Bar[1],x1Bar[2])
%points(x2Bar[1],x2Bar[2])
%points(xBar[1],xBar[2])
%segments(x1Bar[1],x1Bar[2],x2Bar[1],x2Bar[2],col='blue')
%text(x1Bar[1]+.15,x1Bar[2]+.15,expression(bar(X)[1]))
%text(x2Bar[1]+.15,x2Bar[2]+.15,expression(bar(X)[2]))
%text(xBar[1]+.15,xBar[2]+.15,expression(bar(X)))
%\end{blockcode}
%\end{frame}

\begin{frame}[fragile]
\frametitle{Can we just maximize $B$?}
Sadly, no.  Just like $W$ can be made arbitrarily small, $B$ will always be increasing
with increasing $K$.
\begin{figure}[h!]
  \centering
  \includegraphics[width=2.5in,trim=0 0 0 0,clip]{../figures/clusteringKmeansMaximizeB.pdf}  
\end{figure}
\end{frame}

\begin{frame}[fragile]
\frametitle{$CH$ index}
Ideally, we would like our cluster assignment to \alo{simultaneously} have small $W$ and large $B$.
\vsp

This is the idea behind \alo{$CH$ index}.  For clustering assignments coming from $K$
clusters, we record $CH$ score:
\[
CH(K) = \frac{B(K)/(K-1)}{W(K)/(n-K)} 
\]
To choose $K$, pick some maximum number of clusters to be considered ($K_{\max} = 20$,
for example)  and choose the value of $K$ that
\[
\hat K = \argmax_{K \in \{ 2,\ldots, K_{\max} \}} CH(K).
\]

\textcolor{red}{Note:} $CH$ is undefined for $K =1$.

\vsp
\script{Calinski, Harabasz (1974)}
\end{frame}

\begin{frame}[fragile]
\frametitle{$CH$ index}
\begin{blockcode}
ch.index = function(x,kmax,iter.max=100,nstart=10,
                      algorithm="Lloyd")
{
  ch = numeric(length=kmax-1)
  n = nrow(x)
  for (k in 2:kmax) {
    a = kmeans(x,k,iter.max=iter.max,nstart=nstart,
              algorithm=algorithm)
    w = a$tot.withinss
    b = a$betweenss
   ch[k-1] = (b/(k-1))/(w/(n-k))
  }
  return(list(k=2:kmax,ch=ch))
}
\end{blockcode}
\end{frame}

\begin{frame}[fragile]
\frametitle{A simulated example}
\begin{blockcode}
x = matrix(rnorm(50*2),ncol=2)
x[1:25,1] = x[1:25,1] + 3
x[1:25,2] = x[1:25,2] -4
\end{blockcode}
\vsp

We want to cluster this data set using K-means with $K$ chosen via $CH$ index.
\end{frame}

\begin{frame}[fragile]
\frametitle{$CH$ plot}
\begin{figure}[h!]
  \centering
  \includegraphics[width=3in,trim=0 0 0 0,clip]{../figures/clusteringKmeansCH.pdf}  
\end{figure}
\end{frame}

\begin{frame}[fragile]
\frametitle{Corresponding solution}
\begin{figure}[h!]
  \centering
  \includegraphics[width=3in,trim=0 0 0 0,clip]{../figures/clusteringKmeansCHoutcome.pdf}  
\end{figure}
\end{frame}

%\begin{frame}[fragile]
%\frametitle{Alternate approach: Gap statistic}
%While true that $W(K)$ keeps dropping in $K$, \alo{how much it drops} might be informative.
%
%\vsp
%The \alo{gap statistic} is based on this idea.  We compare the observed within-cluster variation
%$W(K)$ to the within-cluster variation we would observe if the data were uniformly distributed $W_{unif}(K)$.
%
%\[
%Gap(K) = \log W(K) - \log W_{unif}(K)
%\]
%After simulating many $\log W_{unif}(K)$, we compute its standard deviation $s(K)$.  Then, we choose
%$K$ by
%\[
%\hat K = \min\{ K \in \{1,\ldots,K_{\max} \} : Gap(K) \geq Gap(K+1) - s(K+1) \}.
%\]
%\script{Tibshirani et al. (2001)}
%\end{frame}

\transitionSlide{Hierarchical clustering}
\begin{frame}
\frametitle{From $K$-means to hierarchical clustering}
Recall two properties of \alg{$K$-means} clustering
\begin{enumerate}
\item It fits exactly $K$ clusters.
\item Final clustering assignments depend on the chosen initial cluster centers.
\end{enumerate}
\vsp

Alternatively, we can use \alg{hierarchical clustering}.  This has the advantage that
\begin{enumerate}
\item No need to choose the number of clusters before hand.
\item There is no random component (nor choice of starting point).
\end{enumerate}
\end{frame}
\begin{frame}
\frametitle{From $K$-means to hierarchical clustering}

There is a catch: we need to choose a way to measure the distance between clusters, called the \alg{linkage}.

\vsp
Given the linkage, hierarchical clustering produces a sequence of clustering assignments.  

\vsp
At one end, all points are in their \alo{own} cluster.  

\vsp
At the other, all points are in \alo{one} cluster.  

\vsp
In the middle, there are \alo{nontrivial} solutions.
\end{frame}

%
%\begin{frame}
%\frametitle{Agglomerative vs. divisive}
%Two types of hierarchical clustering algorithms
%\vsp
%\begin{table}
%\begin{tabular}{ll}
%\smallCapGreen{Agglomerative:} & Start with each point in its own cluster. \\
%                                                             & Merge until all in same cluster.\\
%(ie: top-down)                                    & (think of forward selection) \\
%\hline
%\smallCapGreen{Divisive:}             & Until every point is assigned to its own \\
%(ie: bottom-up)                                  & cluster, repeatedly split the group into two \\
%                                                             & parts that result in the biggest dissimilarity \\
%                                                             & (think of backwards selection).
%\end{tabular}
%\end{table}
%Agglomerative methods are simpler, so we'll focus on them.  You'll have to read about
%divisive methods on your own.
%\end{frame}

\begin{frame}
\frametitle{Agglomerative example}
Given these data points, an agglomerative algorithm \alo{might} decide on the following clustering sequence:

\script{\smallCapGreen{Important:} Different choices of linkage would result in different solutions}
\begin{columns}[T]
\begin{column}{.48\textwidth}
\begin{figure}[h!]
  \centering
  \includegraphics[width=2.3in,trim=0 0 0 55,clip]{../figures/clusteringHierAgglomExample.pdf}
\end{figure}
\end{column}
\begin{column}{.55\textwidth}
\vsp
\begin{enumerate}
\item $\{1\}, \{2\}, \{3\}, \{4\}, \{5\}, \{6\}, \{7\}$
\item $\{1,2\}, \{3\}, \{4\}, \{5\}, \{6\}, \{7\}$
\item $\{1,2\}, \{3\}, \{5\}, \{4,7\}$
\item $\{1,2,6\}, \{3\}, \{5\}, \{4,7\}$
\item $\{1,2,4,6,7\}, \{3\}, \{5\}$
\item $\{1,2,3,4,6,7\},  \{5\}$
\item $\{1,2,3,4,5,6,7\}$
\end{enumerate}
\end{column}
\end{columns}
\end{frame}
%
%\begin{frame}
%\frametitle{Reminder: What's a dendrogram?}
%We encountered dendrograms when we talked about classification and regression trees.
%\vsp
%
%\alg{Dendrogram:} A convenient graphic to display a hierarchical sequence of clustering
%assignments.  This is simply a tree where:
%\begin{itemize}
%\item Each branch represents a group 
%\item Each leaf (terminal) node is a singleton 
%
%\script{ie: a group containing a single data point}
%\item The root node is a group containing the whole data set
%\item Each internal node as two daughter nodes (children), representing the groups that were merged to form it.
%\end{itemize}
%
%\vsp
%Remember: the choice of \alo{linkage} determines how we measure dissimiliarity between groups.
%
%\vsp
%Each internal node is drawn at a height proportional to the dissimiliarity between its two daughter nodes.
%\end{frame}

\begin{frame}
%\frametitle{Agglomerative example}
We can also represent the sequence of clustering assignments as a \alo{dendrogram}
\begin{figure}[h!]
  \centering
  \includegraphics[width=2.3in,trim=0 0 0 55,clip]{../figures/clusteringHierAgglomExample.pdf}
  \includegraphics[width=2.3in,trim=0 40 0 55,clip]{../figures/clusteringHierAgglomExampleDend.pdf}  
\end{figure}
Note that cutting the dendrogram horizontally partitions the data points into clusters
\end{frame}



\begin{frame}
\frametitle{Back to the example}
\begin{figure}[h!]
  \centering
  \includegraphics[width=2.3in,trim=0 0 0 55,clip]{../figures/clusteringHierAgglomExample.pdf}
  \includegraphics[width=2.3in,trim=0 40 0 55,clip]{../figures/clusteringHierAgglomExampleDend.pdf}  
\end{figure}
For instance, the linkage distance between the cluster $\{4,7\}$ and the cluster $\{1,2,6\}$ is about .65.
\end{frame}

\begin{frame}
\frametitle{Linkages}
Notation: Define $X_1,\ldots, X_n$ to be the data

\vsp
Let the \alg{dissimiliarities} be $d_{ij}$ between each pair $X_i, X_j$

\vsp
At any level, clustering assignments can be expressed by sets $G = \{ i_1, i_2, \ldots, i_r\}$. given the \alo{indices}
of points in this group.  Define $|G|$ to be the size of $G$.  

\vsp
\alg{Linkage:}  The function $d(G,H)$ that takes two groups $G,H$ and returns the \alo{linkage} distance between them.

\vsp 
Agglomerative clustering, given the linkage:
\begin{itemize}
\item Start with each point in its own group
\item Until there is only one cluster, repeatedly merge the two groups $G,H$ that minimize $d(G,H)$.
\end{itemize}
\end{frame}

\begin{frame}
\frametitle{Single linkage}
In \alg{single linkage} (a.k.a nearest-neighbor linkage), the linkage distance between $G,H$ is the smallest dissimilarity 
between two points in different groups:
\[
d_{\textrm{single}}(G,H) = \min_{i \in G, \, j \in H} d_{ij}
\]
\begin{columns}[T]
\begin{column}{.52\textwidth}
\vsp

\smallCapGreen{Example:} There are two clusters $G$ and $H$ (\alr{red} and \alb{blue}).  \\ The single linkage
score 

\script{i.e. \alo{$d_{\textrm{single}}(G,H)$}} 

is the dissimilarity between the closest pair 

\script{length of black line segment}
\end{column}
\begin{column}{.48\textwidth}
\begin{figure}[h!]
  \centering
  \includegraphics[width=2.3in,trim=0 0 0 55,clip]{../figures/clusteringHierAgglomExampleSingle.pdf}
\end{figure}
\end{column}
\end{columns}
\end{frame}

\begin{frame}
\frametitle{Single linkage example}
Here $n = 60$, $X_i \in \mathbb{R}^2$, $d_{ij} = || X_i - X_j||_2$.  Cutting the tree at $h = 0.8$ gives the cluster
assignments marked by colors
\begin{figure}[h!]
  \centering
  \includegraphics[width=2.1in,trim=0 30 0 55,clip]{../figures/clusteringHierAgglomExampleSingleSimDend.pdf}
  \includegraphics[width=2.1in,trim=0 0 0 55,clip]{../figures/clusteringHierAgglomExampleSingleSimScatter.pdf}  
\end{figure}
\alo{Cut interpretation:} For each point $X_i$, there is another point $X_j$ in the same cluster with $d_{ij} \leq 0.8$
(assuming more than 1 point in cluster).  Also, no points in different clusters are closer than 0.8.
\end{frame}

\begin{frame}
\frametitle{Complete linkage}
In \alg{complete linkage} (i.e. farthest-neighbor linkage), linkage distance between $G,H$ is the \alo{largest} 
dissimilarity between two points in different clusters:
\[
d_{\textrm{complete}}(G,H) = \max_{i \in G,\, j \in H} d_{ij}.
\]
\begin{columns}[T]
\begin{column}{.52\textwidth}
\vsp

%Example: There are two clusters (red and blue).  The complete linkage
%score ($d_{\textrm{complete}}(G,H)$) is the distance between the \alo{farthest} pair (black line segment).

\smallCapGreen{Example:} There are two clusters $G$ and $H$ (\alr{red} and \alb{blue}).  \\ The complete linkage
score 

\script{i.e. \alo{$d_{\textrm{complete}}(G,H)$}} 

is the dissimilarity between the \alo{farthest} pair 

\script{length of black line segment}
\end{column}
\begin{column}{.48\textwidth}
\begin{figure}[h!]
  \centering
  \includegraphics[width=2.3in,trim=0 0 0 55,clip]{../figures/clusteringHierAgglomExampleComplete.pdf}
\end{figure}
\end{column}
\end{columns}
\end{frame}

\begin{frame}
\frametitle{Complete linkage example}
Same data as before.  Cutting the tree at $h= 3.5$ gives the clustering assignment
\begin{figure}[h!]
  \centering
  \includegraphics[width=2.1in,trim=0 30 0 55,clip]{../figures/clusteringHierAgglomExampleCompleteSimDend.pdf}
  \includegraphics[width=2.1in,trim=0 0 0 55,clip]{../figures/clusteringHierAgglomExampleCompleteSimScatter.pdf}  
\end{figure}
\alo{Cut interpretation:} For each point $X_i$, every other point $X_j$ in the same cluster has $d_{ij} \leq 3.5$.

\end{frame}

\begin{frame}
\frametitle{Average linkage}
In \alg{average linkage}, the linkage distance between $G,H$ is the \alo{average} 
dissimilarity over all points in different clusters:
\[
d_{\textrm{average}}(G,H) = \frac{1}{|G| \cdot |H| }\sum_{i \in G, \,j \in H} d_{ij}.
\]
\begin{columns}[T]
\begin{column}{.52\textwidth}
\vsp

%Example: There are two clusters (red and blue).  The average linkage
%score ($d_{\textrm{average}}(G,H)$) is the average distance between all the points in different clusters.

\smallCapGreen{Example:} There are two clusters $G$ and $H$ (\alr{red} and \alb{blue}).  \\ The average linkage
score 

\script{i.e. \alo{$d_{\textrm{average}}(G,H)$}} 

is the average dissimilarity between \alo{all} points in different clusters

\script{average of lengths of colored line segments}

\end{column}
\begin{column}{.48\textwidth}
\begin{figure}[h!]
  \centering
  \includegraphics[width=2.3in,trim=0 0 0 55,clip]{../figures/clusteringHierAgglomExampleAverage.pdf}
\end{figure}
\end{column}
\end{columns}
\end{frame}

\begin{frame}
\frametitle{Average linkage example}
Same data as before.  Cutting the tree at $h= 1.75$ gives the clustering assignment
\begin{figure}[h!]
  \centering
  \includegraphics[width=2.1in,trim=0 30 0 55,clip]{../figures/clusteringHierAgglomExampleAverageSimDend.pdf}
  \includegraphics[width=2.1in,trim=0 0 0 55,clip]{../figures/clusteringHierAgglomExampleAverageSimScatter.pdf}  
\end{figure}
\alo{Cut interpretation:} ??

\end{frame}

\begin{frame}
\frametitle{Common properties}
Single, complete, and average linkage share the following:
\begin{itemize}
\item They all operate on the dissimilarities $d_{ij}$.  This means that the points we are clustering can be
quite general (number of mutations on a genome, polygons, faces, whatever).
\item Running agglomerative clustering with any of these linkages produces a dendrogram with no \alo{inversions}.
\end{itemize}
\vsp

No inversions means that the linkage distance between merged clusters only \alo{increases} as we run
the algorithm.

\vsp
In other words, we can draw a proper dendrogram, where the height of a parent is always higher than the height
of either daughter.

\script{We'll return to this again shortly}
\end{frame}


\begin{frame}
\frametitle{Shortcomings of single and complete linkage}
Single and complete linkage have practical problems:
\begin{table}
\begin{tabular}{ll}
\smallCapGreen{Single linkage:} & Often suffers from \alo{chaining}, that is, \\
& we only need a single pair of  \\
& points to be close to merge two clusters. \\
 & Therefore, clusters can be too spread out  \\
&  and not compact enough. \\
\smallCapGreen{Complete linkage:} & Often suffers from \alo{crowding}, that is, \\
& a point can be closer to points in \\
& other clusters than to points in its own  \\
& cluster.  Therefore, the clusters are \\ 
& compact, but not far enough apart.
\end{tabular}
\end{table}
Average linkage tries to strike a balance between these two.
\end{frame}



\begin{frame}
\frametitle{Example of chaining and crowding}
\begin{table}
\begin{tabular}{cc}
  \includegraphics[width=1.6in,trim=0 10 0 55,clip]{../figures/clusteringHierAgglomExampleSingleSimScatter.pdf}  &
  \includegraphics[width=1.6in,trim=0 10 0 55,clip]{../figures/clusteringHierAgglomExampleCompleteSimScatter.pdf}   \\
  Single linkage & Complete linkage \\
  \includegraphics[width=1.6in,trim=0 10 0 55,clip]{../figures/clusteringHierAgglomExampleAverageSimScatter.pdf}   & \\
  Average linkage 
\end{tabular}
\end{table}

\end{frame}

\begin{frame}
\frametitle{Shortcomings of average linkage}
Average linkage isn't perfect.
\begin{itemize}
\item It isn't clear what properties the resulting clusters have when we cut an average linkage tree.
\item Results of average linkage clustering can change with a monotone increasing transformation of the
dissimilarities (that is, if we changed the distance, but maintained the ranking of the distances, the cluster
solution could change).
\end{itemize}
Neither of these problems afflict single or complete linkage.
\end{frame}

\begin{frame}
\frametitle{Example of monotone increasing problem}
\begin{table}[h!]
  \centering
\begin{tabular}{ccc}
\parbox[c][.5cm]{1.5cm}{\alo{Average} \vvvsp \vvvsp \vvvsp \vvvsp} 
&  
\includegraphics[width=1.55in,trim=0 0 0 55,clip]{../figures/clusteringHierAgglomExampleAverageSimScatterUnSq.pdf}   
&
\includegraphics[width=1.55in,trim=0 0 0 55,clip]{../figures/clusteringHierAgglomExampleAverageSimScatterSq.pdf}     
\\ 
\parbox[c][.5cm]{1.5cm}{\alo{Single} \vvvsp \vvvsp \vvvsp \vvvsp} 
&  
\includegraphics[width=1.55in,trim=0 0 0 55,clip]{../figures/clusteringHierAgglomExampleSingleSimScatterUnSq.pdf}   
&
\includegraphics[width=1.55in,trim=0 0 0 55,clip]{../figures/clusteringHierAgglomExampleSingleSimScatterSq.pdf}        
\\
& Left: $d_{ij} = ||X_i - X_j||_2$ &  Right: $d_{ij} = ||X_i - X_j||_2^2$ 
\end{tabular}
\end{table}
\end{frame}

\begin{frame}[fragile]
\frametitle{Hierarchical agglomerative clustering in \alr{R}}
The function {\tt hclust} in base \alr{R} performs the necessary computations.  E.g.
\begin{blockcode}
Delta = dist(x)
out.average = hclust(Delta,method='average')
plot(out.average)
\end{blockcode}
\end{frame}

\begin{frame}
\frametitle{Recap}
\smallCapGreen{Hierarchical agglomerative clustering: } Start with all data points in
their own groups, and repeatedly merge groups, based on linkage
function. Stop when points are in one group (this is agglomerative;
there is also divisive)
\vsp

This produces a sequence of clustering assignments, visualized by a
dendrogram (i.e., a tree). Each node in the tree represents a group,
and its height is proportional to the linkage distance of its daughters

\vsp
Three most common linkage functions: \alo{single}, \alo{complete}, \alo{average}
linkage. Single linkage measures the least dissimilar pair between
groups, complete linkage measures the most dissimilar pair,
average linkage measures the average dissimilarity over all pairs
\vsp

Each linkage has its strengths and weaknesses
\end{frame}

\begin{frame}[fragile]
\frametitle{Careful Example}
\begin{columns}[T]
\begin{column}{.65\textwidth}

\begin{figure}[h!]
  \centering
  \includegraphics[width=3.2in,trim=0 0 0 0,clip]{../figures/clusteringHierLinkageStepByStep.pdf}
\end{figure}
\end{column}
\begin{column}{.35\textwidth}
\vsp
\vsp

Distance matrix ($\Delta$)
%\begin{blockcode}
\tiny
\begin{alltt}
     1    2    3    4    5    6    7
1 0.00 0.37 0.49 0.61 0.96 1.24 1.42
2 0.37 0.00 \textcolor{red}{0.12} 0.64 0.98 1.44 1.24
3 0.49 \textcolor{red}{0.12} 0.00 0.65 0.97 1.48 1.15
4 0.61 0.64 0.65 0.00 \textcolor{blue}{0.36} 0.85 0.90
5 0.96 0.98 0.97 \textcolor{blue}{0.36} 0.00 0.71 0.72
6 1.24 1.44 1.48 0.85 0.71 0.00 1.39
7 1.42 1.24 1.15 0.90 0.72 1.39 0.00
\end{alltt}
%\end{blockcode}

\begin{table}
\begin{tabular}{ll}
\multicolumn{2}{l}{(All Merging $\{1\}$ and $\{2,3\}$)} \\
\smallCapGreen{Single}: & 0.37 \\
\smallCapGreen{Complete}: & 0.49 \\
\smallCapGreen{Average}: & (0.37 + 0.49)/2 = 0.43
\end{tabular}
\end{table}


\begin{table}
\begin{tabular}{ll}
\multicolumn{2}{l}{(Next Agglomeration)} \\
\smallCapGreen{Single}: & Merging $\{4,5\}$ \& $\{1,2,3\}$ \\
                                              & 0.61 \\
\smallCapGreen{Complete}: & Merging $\{4,5\}$ \& $\{6\}$ \\
                                              & 0.85 \\
\smallCapGreen{Average}: & Merging $\{4,5\}$ \& $\{6\}$ \\
                                              & (0.85+0.71)/2 = 0.78
\end{tabular}
\end{table}


\end{column}
\end{columns}

\end{frame}

\begin{frame}
\frametitle{Another linkage}
\alo{Centroid linkage} is a commonly used and relatively new approach.  
Assume 
\begin{itemize}
\item $X_i \in \mathbb{R}^p$
\item $d_{ij} = ||X_i - X_j||_2^2$
\end{itemize}
\vsp

Let $\overline{X}_G$ and $\overline{X}_H$ denote group averages for $G,H$. Then
\[
d_{\textrm{centroid}} = ||\overline{X}_G - \overline{X}_H||_2^2
\]

\begin{columns}[T]
\begin{column}{.48\textwidth}
\vsp
\vsp
Example: There are two clusters (red and blue).  The centroid linkage
score ($d_{\textrm{centroid}}(G,H)$) is the distance$^2$ between the centroids (black line segment).
\end{column}
\begin{column}{.48\textwidth}
\begin{figure}[h!]
  \centering
  \includegraphics[width=2.03in,trim=0 0 0 55,clip]{../figures/clusteringHierAgglomExampleCentroid.pdf}
\end{figure}
\end{column}
\end{columns}
\end{frame}

\begin{frame}
\frametitle{Centroid linkage}
Centroid linkage is
\begin{itemize}
\item  ... quite intuitive
\item ... widely used
\item ... nicely analogous to $K$-means.
\item ... very related to average linkage (and much, much faster)
\end{itemize}
However, it has a very unsavory feature: \alo{inversions}.

\vsp
An inversion is when an agglomeration doesn't reduce the linkage distance.
\end{frame}

\begin{frame}
\frametitle{Centroid linkage example}
Same data as before.  We can't look at cutting the tree, but we can still look at a 3 cluster solution.
\begin{figure}[h!]
  \centering
  \includegraphics[width=2.1in,trim=0 30 0 55,clip]{../figures/clusteringHierAgglomExampleCentroidSimDend.pdf}
  \includegraphics[width=2.1in,trim=0 0 0 55,clip]{../figures/clusteringHierAgglomExampleCentroidSimScatter.pdf}  
\end{figure}
\alo{Cut interpretation:} Even if there are no inversions, there still is no cut interpretation.

\end{frame}


\begin{frame}[fragile]
\frametitle{Careful Example: Steps 1,2,3}

\begin{figure}[h!]
  \centering
  \includegraphics[width=3.2in,trim=0 250 0 0,clip]{../figures/clusteringHierLinkageStepByStepCentroid.pdf}
\end{figure}

\begin{columns}[T]
\begin{column}{.45\textwidth}
Distance matrix ($\Delta$)
%\begin{blockcode}
\tiny
\begin{alltt}
     1    2    3    4    5    6    7
1 0.00 0.40 0.35 1.62 1.20 2.16 3.67
2 0.40 0.00 \textcolor{blue}{0.34} 1.96 0.37 2.54 3.66
3 0.35 \textcolor{blue}{0.34} 0.00 0.68 0.42 1.05 1.94
4 1.62 1.96 0.68 0.00 1.45 \textcolor{red}{0.04} 0.46
5 1.20 0.37 0.42 1.45 0.00 1.86 2.35
6 2.16 2.54 1.05 \textcolor{red}{0.04} 1.86 0.00 0.27
7 3.67 3.66 1.94 0.46 2.35 0.27 0.00
\end{alltt}
\script{This is squared Euclidean distance}
\end{column}
\begin{column}{.55\textwidth}
\textcolor{red}{Centroid(4,6) $=$ (1.68,0.76)} \\
\textcolor{blue}{Centroid(2,3) $=$ (0.58,1.25)} \\
%\textcolor{green}{Centroid distance$^2$: 1.46} \\
\end{column}
\end{columns}
\end{frame}

\begin{frame}[fragile]
\frametitle{Careful Example: Step 4}

\begin{figure}[h!]
  \centering
  \includegraphics[width=3.2in,trim=0 250 0 0,clip]{../figures/clusteringHierLinkageStepByStepCentroidPart2.pdf}
\end{figure}

\begin{columns}[T]
\begin{column}{.45\textwidth}
Distance matrix ($\Delta$)
%\begin{blockcode}
\tiny
\begin{alltt}
     1    2    3    4    5    6    7
1 0.00 0.40 0.35 1.62 1.20 2.16 3.67
2 0.40 0.00 \textcolor{blue}{0.34} 1.96 0.37 2.54 3.66
3 0.35 \textcolor{blue}{0.34} 0.00 0.68 0.42 1.05 1.94
4 1.62 1.96 0.68 0.00 1.45 0.04 0.46
5 1.20 0.37 0.42 1.45 0.00 1.86 2.35
6 2.16 2.54 1.05 0.04 1.86 0.00 0.27
7 3.67 3.66 1.94 0.46 2.35 0.27 0.00
\end{alltt}
\script{This is squared Euclidean distance}
\end{column}
\begin{column}{.55\textwidth}
Which one gets merged?
\pause

\script{$\{1\}$ and $\{2,3\}$}
\pause
\tiny
\begin{alltt}
method = 'centroid'
out = hclust(Delta,method=method)
rect.hclust(out,k=5,
            border=c('white','red','white','white','blue'))
\end{alltt}
\end{column}
\end{columns}
\end{frame}

%\begin{frame}[fragile]
%\frametitle{Careful Example: Step 5}
%
%\begin{figure}[h!]
%  \centering
%  \includegraphics[width=3.2in,trim=0 250 0 0,clip]{../figures/clusteringHierLinkageStepByStepCentroidPart3.pdf}
%\end{figure}
%
%\begin{columns}[T]
%\begin{column}{.35\textwidth}
%Distance matrix ($\Delta$)
%%\begin{blockcode}
%\tiny
%\begin{alltt}
%     1    2    3    4    5    6    7
%1 0.00 0.28 1.16 0.69 0.90 1.21 1.68
%2 0.28 0.00 0.88 0.76 0.96 1.04 1.42
%3 1.16 0.88 0.00 1.41 1.54 1.00 0.74
%4 0.69 0.76 1.41 0.00 0.21 0.89 1.62
%5 0.90 0.96 1.54 0.21 0.00 0.89 1.67
%6 1.21 1.04 1.00 0.89 0.89 0.00 0.82
%7 1.68 1.42 0.74 1.62 1.67 0.82 0.00
%\end{alltt}
%\end{column}
%\begin{column}{.65\textwidth} 
%Merge $\{3,7\}$ and $\{6\}$.
%\end{column}
%\end{columns}
%\end{frame}
%

\begin{frame}[fragile]
\frametitle{Linkages summary}

\begin{table}
\begin{tabular}{l|cccc}
 & \parbox{.93in}{No inversions?} & \parbox{.9in}{Unchanged w/ monotone  transformation?} & \parbox{.55in}{Cut interpretation?} & Notes \\
 \hline \\
\smallCapGreen{Single} & 
\textcolor{blue}{$\checkmark$} & \textcolor{blue}{$\checkmark$}  & \textcolor{blue}{$\checkmark$} & 
\textcolor{red}{chaining} \\
\hline \\
\smallCapGreen{Complete} &
\textcolor{blue}{$\checkmark$} & \textcolor{blue}{$\checkmark$} & \textcolor{blue}{$\checkmark$} & 
\textcolor{red}{crowding}\\
\hline \\
\smallCapGreen{Average} &
\textcolor{blue}{$\checkmark$} & \textcolor{red}{$\text{X}$} & \textcolor{red}{$\text{X}$} & \\
\hline \\
\smallCapGreen{Centroid} &
\textcolor{red}{$\text{X}$} & \textcolor{red}{$\text{X}$} & \textcolor{red}{$\text{X}$} & 
\textcolor{red}{inversions} \\
\hline 
\end{tabular}
\end{table}
Final notes:

\begin{itemize}
\item None of this helps determine what is the \alo{best} linkage
\item Use the linkage that seems the most appropriate for the types of clusters you want to get
\end{itemize}
\end{frame}


\begin{frame}[fragile]
\frametitle{Designing a clever radio system}
We have a lot of songs and dissimilarity scores between them ($d_{ij}$)

\vsp
We want to build a clever radio system that takes a song specified by the user and produces
a song of the ``same'' type

\vsp
We ask the user how ``risky'' he or she wants to be

\begin{figure}
\centering
  \includegraphics[width=1.5in,trim=0 0 0 0,clip]{../figures/pandora.pdf} 
\end{figure}

How can we use hierarchical clustering and with what linkage?
\end{frame}


\begin{frame}[fragile]
\frametitle{Linkages summary: Cut interpretations}
Suppose we cut the tree at height $h = 1$.
\begin{table}
\begin{tabular}{l|l}
\hline \\
\smallCapGreen{Single} & 
 For each point $X_i$, there is another point $X_j$ in the  \\ 
 & same cluster with $d_{ij} \leq 1$ (assuming more than \\ 
 & 1 point in  cluster).  Also, no points in different clusters \\ 
 & are closer than 1. \\
 \\
\hline \\
\smallCapGreen{Complete} &
For each point $X_i$, every other point $X_j$ in the same \\
& cluster has $d_{ij} \leq 1$. \\
\\
\hline 
\end{tabular}
\end{table}
\end{frame}

\begin{frame}[fragile]
\frametitle{Data analysis example}
Diffuse large B-cell lymphoma (DLBCL) is the most common type of non-Hodgkin's lymphoma

\vsp
It is clinically heterogeneous: 
\begin{itemize}
\item 40\% of patients respond well
\item 60\% of patients succumb to the disease
\end{itemize}
\vsp
The researchers propose that this difference is due to unrecognized molecular heterogeneity in the tumors

\vsp
We examine the extent to which genomic-scale gene expression profiling can further the understanding of 
B-cell malignancies.
\end{frame}
\begin{frame}[fragile]
\frametitle{Data analysis example}
Here, we have gene expression data at 2,000 genes for 62 cancer cells.

\vsp 
There are 3 cancer diagnoses: FL, CLL, DLBCL.  Each corresponds to a type of malignant lymphoma.

\vsp 
We want to use hierarchical clustering to understand this data set better.

\begin{blockcode}
load('../data/alizadeh.RData')

genesT  = alizadeh$x
genes   = t(genesT)
Yfull   = alizadeh$type
Y       = as.vector(Yfull)
Y[Yfull == "DLBCL-A"] = 'DLBCL'
Y[Yfull == "DLBCL-G"] = 'DLBCL'
Y       = as.factor(Y)
dist.mat    = dist(genes)
\end{blockcode}
\end{frame}

\begin{frame}[fragile]
\frametitle{PCA plot}
\begin{columns}[T]
\begin{column}{.65\textwidth}
\begin{figure}
\centering
  \includegraphics[width=2.5in,trim=0 0 0 0,clip]{../figures/clusteringGenesPCA.pdf} 
\end{figure}
\end{column}
\begin{column}{.35\textwidth} 
\vsp
\vsp
\vsp

Two clear groups for FL and CLL

\vsp
DLBCL somewhat appears to be 1 group, but it is much more diffuse.
\end{column}
\end{columns}


\end{frame}

\begin{frame}[fragile]
\frametitle{PCA plot}
\begin{columns}[T]
\begin{column}{.65\textwidth}
\begin{figure}
\centering
  \includegraphics[width=2.5in,trim=0 0 0 0,clip]{../figures/clusteringGenesPCAfull.pdf} 
\end{figure}
\end{column}
\begin{column}{.35\textwidth} 
\vsp
\vsp
\vsp

Here are the two sub-types identified by the researchers

\vsp
Let's look at their results further.
\end{column}
\end{columns}
\end{frame}

\begin{frame}[fragile]
\frametitle{Four hierarchical cluster solutions}
\scriptsize
\begin{table}
\begin{tabular}{cc}
  \includegraphics[width=1.7in,trim=0 40 0 55,clip]{../figures/clusteringGenesSingle.pdf}  &
    \includegraphics[width=1.7in,trim=0 40 0 55,clip]{../figures/clusteringGenesComplete.pdf}  \\
    Single & Complete \\
  \includegraphics[width=1.7in,trim=0 40 0 55,clip]{../figures/clusteringGenesAverage.pdf}  &
    \includegraphics[width=1.7in,trim=0 40 0 55,clip]{../figures/clusteringGenesCentroid.pdf}  \\
  Average & Centroid 
\end{tabular}
\end{table}
\end{frame}

\begin{frame}[fragile]
\frametitle{Complete linkage: A closer look}
\begin{columns}[T]
\begin{column}{.55\textwidth}
\begin{figure}
\centering
  \includegraphics[width=2.5in,trim=0 0 0 0,clip]{../figures/clusteringGenesCompleteRect.pdf} 
\end{figure}
\end{column}
\begin{column}{.45\textwidth} 
\vsp
\vsp
\vsp

\begin{blockcode}
out.com = hclust(dist.mat,
     method='complete')
plot(out.com,xlab='',
   main='',labels=Y)
rect.hclust(out.com,k=12)

out.cut = cutree(out.com,
         k=12)
\end{blockcode}
Notice that FL and CLL are distinctly grouped, while there are many clusters inside the DLBCL type.
\end{column}
\end{columns}
\end{frame}
%
%\begin{frame}[fragile]
%\frametitle{Cluster methodology}
%In the paper, the researchers used hierarchical clustering to find cancer subtypes
%\vsp
%
%They used a variation on the methodology we have discussed
%
%\vsp 
%For a distance, they used correlation
%
%\vsp
%They used a centroid linkage (oddly, they do not encounter any inversions...)
%\end{frame}

%
%\begin{frame}[fragile]
%\frametitle{Full microarray}
%\begin{columns}[T]
%\begin{column}{.6\textwidth}
%\begin{figure}
%\centering
%  \includegraphics[width=2.7in,trim=20 0 30 80,clip]{../figures/geneNaturefigure1.pdf} 
%\end{figure}
%\end{column}
%\begin{column}{.4\textwidth} 
%\scriptsize
%\vsp
%\vsp
%
%\vsp
%Each row is a cDNA clone (essentially a gene) and each column
%is a mRNA sample (essentially an observation).
%
%\vsp
%The green/red colors on the microarray indicate mRNA expression relative to a reference.
%
%\vsp
%The color coding for labels was added {\it after} the clustering.  The non-cancer related colors are 
%experimental controls (other tissues).
%
%\vsp
%Both FL and CLL are near resting B-cells (\alo{indolence})
%
%\vsp
%DLBCL is highly expressed in the area where proliferation is controlled.  However, there is
%substantial variation
%\end{column}
%\end{columns}
%\end{frame}
%
%\begin{frame}[fragile]
%\frametitle{Subset of microarray}
%\begin{columns}[T]
%\begin{column}{.6\textwidth}
%\begin{figure}
%\centering
%  \includegraphics[width=2.5in,trim=20 0 30 50,clip]{../figures/geneNaturefigure2.pdf} 
%\end{figure}
%\end{column}
%\begin{column}{.4\textwidth} 
%\scriptsize
%\vsp
%\vsp
%
%\vsp
%The variable presence of T lymphocytes in DLBCL biopsies was
%readily discernible by a T-cell gene expression signature
%
%\vsp
%Again, notice the difference in expression for DLBCL in proliferation region
%
%\vsp
%Likewise, FL and CLL are unexpressed in proliferation region.
%
%\vsp
%FL is highly expressed in Germinal Center B while  CLL is not.
%
%\vsp
%Some of the DLBCL is equally expressed in Lymph Node biology as
%normal lymph node/tonsil cells (grey). 
%
%\end{column}
%\end{columns}
%\end{frame}
%
%
%\begin{frame}[fragile]
%\frametitle{Creating sub-types for DLBCL}
%\begin{columns}[T]
%\begin{column}{.6\textwidth}
%\begin{figure}
%\centering
%  \includegraphics[width=2.5in,trim=20 0 30 40,clip]{../figures/geneNaturefigure3.pdf} 
%\end{figure}
%\end{column}
%\begin{column}{.4\textwidth} 
%\scriptsize
%\vsp
%\vsp
%
%The vertical blue/orange bars indicate high expression levels associated with GC B-like
%and Activated B-like DLBCL, respectively.
%
%\begin{itemize}
%\item[\textbf{\textcolor{black}{A}}] The DLBCL type clustered into two groups
%based only on genes in Germinal Center B region.
%\item[\textbf{\textcolor{black}{B}}] A subset of the genes (excluding Proliferation,
%T-cell, and Lymph-node) ordered by hierarchical clustering.
%\item[\textbf{\textcolor{black}{C}}] Only looking at the differentially expressed regions from B.
%\end{itemize}
%\end{column}
%\end{columns}
%\end{frame}
%
%\begin{frame}[fragile]
%\frametitle{Comparing sub-types to mortality data}
%\begin{figure}
%\centering
%  \includegraphics[width=3in,trim=20 170 30 170,clip]{../figures/geneNatureFigure5.pdf} 
%\end{figure}
%Survival curve of patients, grouped by...
%\begin{itemize}
%\item[\textbf{\textcolor{black}{A}}] clustered sub-type
%\item[\textbf{\textcolor{black}{B}}] International Prognostication Index (IPI) \\($0 \leq $ IPI $ \leq 2$ vs. 
%$3 \leq $ IPI $ \leq 5$).
%\item[\textbf{\textcolor{black}{C}}] Low clinical risk ($0 \leq $ IPI $ \leq 2$), by clustered sub-type.
%\end{itemize}
%\script{Note: these are known as \alg{Kaplan-Meier plots}}
%\end{frame}
%
%\begin{frame}[fragile]
%\frametitle{Data analysis example: Conclusion}
%The researchers found convincing sub-types of distinct cancers in previously diagnoses
%\vsp
%
%This result is quite interesting:
%\begin{itemize}
%\item \textcolor<1>{greenmain}{Compelling evidence that differential response to cancer treatments is the result (on average) of
%insufficiently differentiated cancer types}
%\item \textcolor<2>{greenmain}{Calls into question fundamental definitions such as what is a disease?}
%\item \textcolor<3>{greenmain}{Provides an alternative method to generate clinical diagnosis categories}
%\end{itemize}
%
%\vsp
%\textcolor<4>{greenmain}{\textcolor<-3>{white}{Conclusion: Qualitative {\it symptoms-based} categorization of disease is inadequate for diagnosis and treatment}}
%\end{frame}
%
%\begin{frame}[fragile]
%\frametitle{Wrap-up}
%Just like in $K$-means, we can use CH-index to pick the number of clusters:
%\begin{blockcode}
%cutree(out.tree,k=K)
%\end{blockcode}
%\vsp
%
%There are many, many other clustering approaches to take: KNN, Factor analysis, level-sets of density
%estimators, mixture of normals, ...
%\end{frame}

\end{document}
