%Preamble
\documentclass{article}
\usepackage{amsmath,amsthm,amssymb,amsfonts}
\usepackage[numbers,square]{natbib}
\usepackage[colorlinks]{hyperref}


\newcommand{\B}{{\tt BibTeX}}



\begin{document}
%Top Matter
\title{Latex lecture}
\author{Darren Homrighausen}
\date{\today}
\maketitle

\section{Bibliography}
There are two main ways to do a bibliography in \LaTeX{}.  I will pejoratively refer to them as the `bad' way and 'good' way\footnote{Feel free to make you're own labels.}.  In either case, the body of your document is the same.  You cite via
\begin{verbatim}
As shown in \cite{HomrighausenMcDonald2013}, \ldots
\end{verbatim}
which looks like:

As shown in \cite{HomrighausenMcDonald2013}, \ldots
\subsection{Bad way}
\begin{verbatim}
\begin{thebibliography}{99}
\bibitem{HomrighausenMcDonald2013} 
Homrighausen, D. and McDonald, D.J.
\emph{Leave-one-out cross-validation is risk consistent for lasso},
Machine Learning (2013)
\end{thebibliography}
\end{verbatim}

\begin{thebibliography}{99}
\bibitem{HomrighausenMcDonald2013} 
Homrighausen, D. and McDonald, D.J.
\emph{Leave-one-out cross-validation is risk consistent for lasso},
Machine Learning (2013)
\end{thebibliography}

\subsubsection{Comments}
This approach has some major drawbacks:
\begin{itemize}
\item You have to have all of your bibitems in the document itself.
\item It is difficult to change the format of your citations.
\end{itemize}
However, it also has one advantage over the good way (I'll say exactly what this is after introducing it).

\subsection{Good way}
A much better way, overall, of doing bibliographies is via \B.  \B needs three extra ingredients: 
\begin{itemize}
\item a `usepackage' statement in the preamble.  
\item an extra file, referred to as a `dot bib' file (ie: .bib)
\item `bibliographystyle' and bibliography statements in the footer.
\end{itemize}

\subsubsection{The usepackage statement}
Here, you need to add something like
\begin{verbatim}
\usepackage[numbers,square]{natbib}
\end{verbatim}
to your preamble.  The `natbib' is built into the \LaTeX{} distribution.  

\subsubsection{The .bib file}
This file contains all the equivalent entries to the bibitem statement from before.  
However, it is expressed in a different manner to help with formatting.  An example entry looks like:
\begin{verbatim}
@article{HomrighausenMcDonald2013a,
	Author = {Homrighausen, D. and McDonald, D.J.},
	Title = {Leave-one-out cross-validation is risk consistent for lasso},
	Journal = {Machine Learning},
	Year = {2013}}
\end{verbatim}

For our example, I have created a file `refs.bib' that contains the relevant entries.

\subsubsection{bibliographystyle and bibliography}
In your footer, you need to add the following:
\begin{verbatim}
\bibliographystyle{elsarticle-num-names}
\bibliography{refs}
\end{verbatim}

These two statements control the formatting of the bibliography and point to the `.bib' file, respectively. 
There are other formatting files you can use (journals
with sometimes give them to you, for instance).  These files are `.bst' files and are the analogue to `.sty' files from \LaTeX{}.
Remember the `.bib. file is called `refs.bib'.



\section{Repositories}
As discussed, we need access to many files to make \LaTeX{} work for academic articles (such as .bib, .bst, .sty, .cls, ...).  We can
put these files in the directory that contains the .tex file and everything will work fine.  

What if we are writing two papers in different directories?  It would be better if we didn't have two copies of everything (version control...).
\subsection{Mac OS}

On Mac OS, the directory `texmf' is the ticket.  Add the following folders to your system (Add the appropriate files to the appropriate directories):

\begin{verbatim}
~/Library/texmf
~/Library/texmf/tex
~/Library/texmf/tex/latex
~/Library/texmf/bibtex
~/Library/texmf/bibtex/bib
~/Library/texmf/bibtex/bst
\end{verbatim}

\subsection{Windows}
It is my understanding that the equivalent part to MixTeX is via the start menu:
\begin{enumerate}
\item Start menu
\item MixTeX
\item Maintenance
\item Settings
\end{enumerate}
From here, you can look at the directories in the search tree and add new ones.

\subsection{Dropbox}
I use several computers.  Dropbox is an easy way to sync them (though there are others).  So, again in the interest of version control, I want only one texmf repository shared among all my computers. Do the following from terminal:
\begin{enumerate}
\item kpsewhich texmf.cnf (finds the appropriate file)
\item emacs /usr/local/texlive/2012/texmf.cnf 
\item There will be line: `TEXMFHOME = ~/Library/texmf'
\item change this to: `TEXMFHOME = ~/Library/texmf:NEWDIRECTORY', where NEWDIRECTORY could be ./Dropbox/texmf..
\end{enumerate}

\section{Miscellaneous}
Here is truly miscellanea that I though might be helpful:
\begin{itemize}
\item- Sometimes your document won't seem to compile.  Try to remove the .aux file and recompile.
\item \begin{verbatim} \DeclareMathOperator*{\argmin}{argmin} \end{verbatim}
\end{itemize}
\end{document}