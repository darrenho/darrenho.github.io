%Preamble
\documentclass{article}
\usepackage{subfig}
\usepackage{graphicx}
\usepackage{amsmath,amsthm,amssymb,amsfonts}
%Ignore for now%%% \usepackage[options]{natbib}
%Ignore for now%%% \renewcommand{\abstractname}{Not abstract}
\usepackage[colorlinks]{hyperref}

\newcommand{\G}{\mathcal{G}}
\renewcommand{\L}[3]{\mathcal{L}_{#1}(#2|#3)}

\begin{document}
%Top Matter
\title{Latex lecture}
\author{Darren Homrighausen}
\date{\today}
\maketitle

\section{What is \LaTeX?}
It is pronouned `la' + 'tech' (the `X' is actually a $\chi$).

\subsection{Summary of what it is/does}
\LaTeX{} is a compiled mark-up language.  The main idea is that you write a document that includes all the formatting instruction and actual text.  This document gets compiled by \LaTeX{} into some device-independent file format, one of the following
\begin{itemize}
\item  .dvi (which isn't used very often)
\item .ps (which is not used very much and is relatively large)
\item .pdf (pretty much the standard)
\end{itemize}

I'll go into some more details on this process later.
\section{Software}

\begin{itemize}
\item For Mac OS X, use \href{http://tug.org/mactex/}{MacTex}.
\item For Linux, you don't need to do anything (it is already installed).
\item For Windows there are two major options:
\begin{itemize}
\item \href{http://miktex.org}{MikTex}.
\item \href{http://www.tug.org/texlive/}{TexLive}.
\end{itemize}
Unfortunately, I haven't used Windows since version 3.11 (and even then, I mostly used the DOS interface).  So, I am essentially useless on trouble shooting problems for either of these programs.  If you have any questions regarding their installation, please
contact our system administrator Zube (Zube@stat.colostate.edu).
\end{itemize}

It is (somewhat) important to note that the software both provide the {\em compiler} and {\em editors}.  The compilers are crucial,
you can choose to use the editor or not.

\subsection{Editors}
\begin{itemize}
\item MacTex comes with TeXShop, which is a nice editor.
\item Alternatively, there is Aquamacs (which is a Mac OS X version of Emacs).   This has a very versatile interface and 
includes the awe-inspiring Emacs interface.
\item Choose whatever editor you want and compile the source directly.
\end{itemize}

\section{The basics of \LaTeX{}}
A basic \LaTeX{} document is composed of several ingredients: Preamble, Top matter, body, and bottom matter.
\subsection{Preamble}
The preamble sets up all of the formatting instructions for you.  Here is an example:

\begin{verbatim}
\documentclass{article}
\usepackage{amsmath,amsthm,amssymb,amsfonts}

\begin{document}
\end{verbatim}

There are three components to a basic document (there are far more advanced things).

\begin{itemize}
\item documentclass: Determines what type of \href{http://en.wikibooks.org/wiki/LaTeX/Document_Structure#Document_classes}{document} you are making.  You will mainly use article or specific classes developed by journals (we will return to this later).
\item usepackage: this is how to load packages that contain functions you want to use.  Here, we have loaded all of the math packages for making the appropriate math symbols.
\item begin\{document\}:  Tells latex to start looking for text.
\end{itemize}

As an aside, you can make \LaTeX{} display things exactly as written (instead of formatting it) by using\footnote{By the way, this is about as meta as I can stand.}:

\begin{verbatim}
begin{verbatim}
...by using\footnote{By the way, this is about as meta as I can stand.}:
end{verbatim}
\end{verbatim}

\subsection{Top matter}
Here, you put all the meta information for the document.

\begin{verbatim}
\title{Example}
\author{Darren Homrighausen}
\date{\today}
\maketitle

\begin{abstract}
This is all so abstract.
\end{abstract}
\end{verbatim}

\subsection{Body}
This is were the things you will see in the compiled document get written.  For example:
\begin{verbatim}
\section{Introduction}
In the beginning...
\end{verbatim}
\subsection{Bottom matter}
Fairly simple:

\begin{verbatim}
%Bibliography stuff

\end{document}
\end{verbatim}
The bibliography stuff we will return to later.  The `end document' statement needs to go at the end of every document.  Anything placed after it will not be compiled.

\section{Math stuff}
The main utility (though far from the only) of \LaTeX{} is in displaying math.  There are three main ways of doing math formatting:

There are two great resources for learning the syntax.  These are the \href{http://en.wikibooks.org/wiki/LaTeX/Mathematics}{general} and \href{http://en.wikibooks.org/wiki/LaTeX/Advanced_Mathematics}{advanced} wikibooks entries and the \href{http://tug.ctan.org/info/lshort/english/lshort.pdf}{not-so-short introduction to \LaTeX}.
\subsection{In-line}
Suppose you want to write the math-equivalent of `let x be a d-dimensional vector', then you could write
\begin{verbatim}
Let $x \in \mathbb{R}^d$ be given
\end{verbatim}
Let $x \in \mathbb{R}^d$ be given

\subsection{Broken-out, single line}
Sometimes, you want to write some math on its own line.  There are two main options for this:
\begin{verbatim}
\begin{equation*}
Y = \mathbb{X}\beta + \sigma Z
\end{equation*}
\end{verbatim}
\begin{equation*}
Y = \mathbb{X}\beta + \sigma Z
\end{equation*}
\begin{verbatim}
\begin{equation}
Y = \mathbb{X}\beta + \sigma Z
\end{equation}
\end{verbatim}
\begin{equation}
Y = \mathbb{X}\beta + \sigma Z
\end{equation}

\subsection{Broken-out, multiple lines}
Other times, you want to write some math, but it takes too much space.  In this case, use
\begin{verbatim}
\begin{align*}
Y_i & = \beta_0 + \beta_1 X_{i1} + \ldots + \beta_1 X_{ip} + \sigma Z_i \\
    & = \mathbb{X}\beta + \sigma Z_i 
\end{align*}
\end{verbatim}
\begin{align*}
Y_i & = \beta_0 + \beta_1 X_{i1} + \ldots + \beta_1 X_{ip} + \sigma Z_i \\
       & = \mathbb{X}\beta + \sigma Z_i \\
\end{align*}

\begin{verbatim}
\begin{align}
Y_i & = \beta_0 + \beta_1 X_{i1} + \ldots + \beta_1 X_{ip} + \sigma Z_i \\
    & = \mathbb{X}\beta + \sigma Z_i \\
\end{align}
\end{verbatim}
\begin{align}
Y_i & = \beta_0 + \beta_1 X_{i1} + \ldots + \beta_1 X_{ip} + \sigma Z_i \\
       & = \mathbb{X}\beta + \sigma Z_i 
\end{align}

\begin{verbatim}
\begin{align}
Y_i & = \beta_0 + \beta_1 X_{i1} + \ldots + \beta_1 X_{ip} + \sigma Z_i \notag \\
    & = \mathbb{X}\beta + \sigma Z_i \\
\end{align}
\end{verbatim}
\begin{align}
Y_i & = \beta_0 + \beta_1 X_{i1} + \ldots + \beta_1 X_{ip} + \sigma Z_i \notag \\
       & = \mathbb{X}\beta + \sigma Z_i 
\end{align}

\subsection{Warning}
It is tempting to use other equation formatting functions in \LaTeX{} (eqnarray, \$\$,..).  Don't due it.  They will someday 
not be supported and it is just bad practice.

\section{Figures}
It is crucial to get the flow of figures into your documents going smoothly.

The first step is to put the graphicx package into the preamble:
\begin{verbatim}
%Preamble
\documentclass{article}
\usepackage{graphicx}
\usepackage{amsmath,amsthm,amssymb,amsfonts}
\end{verbatim}

Let's make a sample figure:
\begin{verbatim}
x = rnorm(100)
y = 10 + 2*x + rnorm(100)

out = lm(y~x)
plot(x,y,xlab='X',ylab='Y')
abline(a = out$coeff[1],b = out$coeff[2])

pdf('regressionPlot.pdf')
plot(x,y,xlab='X',ylab='Y')
abline(a = out$coeff[1],b = out$coeff[2])
dev.off()
\end{verbatim}

We can put this into \LaTeX{}via:
\begin{verbatim}
\begin{figure}[h!]
  \centering
  \includegraphics[width=3in]{./figures/regressionPlot.pdf}
  \caption{Caption for figure.}
  \label{fig:greenRed}
\end{figure}
\end{verbatim}
\begin{figure}[h!]
  \centering
  \includegraphics[width=3in]{./figures/regressionPlot.pdf}
  \caption{Caption for figure.}
  \label{fig:greenRed}
\end{figure}

\subsection{Things not to do (and what to do instead)}
Two common desires when writting papers is to want to put many figures in a single figure environment and to want to include a legend.  The legend one is easy: don't do it.  Just describe the figure in a caption\footnote{This is of course just my opinion/experience.}.
\subsubsection{Multiple figures}
It is tempting to try and include multiple figures in the same device through {\tt R}, like so:
\begin{verbatim}
pdf('regressionPlotMultiple.pdf')
par(mfrow=c(2,2))
plot(x,y,xlab='X',ylab='Y',main='Regression')
abline(a = out$coeff[1],b = out$coeff[2])

plot(x,y - fitted(out),xlab='X',ylab='Y',main='Residuals')
abline(h=0)

plot(x,cooks.distance(out),xlab='X',ylab="Cook's distance",main="Cook's D")
abline(h=0)

Quantiles = y-fitted(out)
qqplot(x,Quantiles,main='QQplot')
abline(a=0,b=1)
dev.off()
\end{verbatim}

\begin{figure}[h!]
  \centering
  \includegraphics[width=3in]{./figures/regressionPlotMultiple.pdf}
  \caption{Putting all figures together with {\tt R}.}
  \label{fig:greenRed}
\end{figure}
I encourage you to \textbf{not} to do this, as you are `locked in' to formatting (you can't change margins, rearrange figures, etc.). 
Instead, there are a couple of better options.  First, generate 4 separate devices (not shown, but it should be clear how to do this).

You can put them into \LaTeX{}via:
\begin{verbatim}
\begin{figure}[h!]
  \centering
  \includegraphics[width=1.5in]{./figures/regressionPlotMultiple1.pdf}
  \includegraphics[width=1.5in]{./figures/regressionPlotMultiple2.pdf} \\
  \includegraphics[width=1.5in]{./figures/regressionPlotMultiple3.pdf}
  \includegraphics[width=1.5in]{./figures/regressionPlotMultiple4.pdf}
  \caption{Caption for figure.}
  \label{fig:greenRed}
\end{figure}
\end{verbatim}

\begin{figure}[h!]
  \centering
  \includegraphics[width=1.5in]{./figures/regressionPlotMultiple1.pdf}
  \includegraphics[width=1.5in]{./figures/regressionPlotMultiple2.pdf} \\
  \includegraphics[width=1.5in]{./figures/regressionPlotMultiple3.pdf}
  \includegraphics[width=1.5in]{./figures/regressionPlotMultiple4.pdf}
  \caption{Caption for figure.}
  \label{fig:greenRed}
\end{figure}

Alternatively, we can use the package `subfig'.
\begin{verbatim}
%Preamble
\documentclass{article}
\usepackage{graphicx}
\usepackage{subfig}
\usepackage{amsmath,amsthm,amssymb,amsfonts}
\end{verbatim}

Now, instead we can put them into  \LaTeX{}via:
\begin{verbatim}
\begin{figure}[h!]
  \centering
  \subfloat[Regression fit]{ \includegraphics[width=1.5in]{./figures/regressionPlotMultiple1.pdf}}
  \subfloat[Residuals plot]{ \includegraphics[width=1.5in]{./figures/regressionPlotMultiple2.pdf} }\\
  \subfloat[Leverage plot]{  \includegraphics[width=1.5in]{./figures/regressionPlotMultiple3.pdf}}
  \subfloat[QQ-plot]{  \includegraphics[width=1.5in]{./figures/regressionPlotMultiple4.pdf}}
  \caption{Caption for figure with subfig.}
  \label{fig:greenRed}
\end{figure}
\end{verbatim}

\begin{figure}[h!]
  \centering
 \subfloat[Regression fit]{ \includegraphics[width=1.5in]{./figures/regressionPlotMultiple1.pdf}}
 \subfloat[Residuals plot]{ \includegraphics[width=1.5in]{./figures/regressionPlotMultiple2.pdf} }\\
 \subfloat[Leverage plot]{  \includegraphics[width=1.5in]{./figures/regressionPlotMultiple3.pdf}}
 \subfloat[QQ-plot]{  \includegraphics[width=1.5in]{./figures/regressionPlotMultiple4.pdf}}
 \caption{Caption for figure with subfig.}
 \label{fig:greenRed}
\end{figure}

\subsection{File Formats for Figures and Compilation}
The contents of this section are becoming less important.  However, it still comes up and it is better to know what you
are getting youself into.  The main idea is that \LaTeX{} was designed to do the following:
\begin{enumerate}
\item The .tex file gets compiled to a .dvi file.
\item The .dvi file gets converted to a .ps file (the unix command is dvips, incidently)
\item The .ps file gets converted to a .pdf
\end{enumerate}
The reason for this is that while Adobe Systems made the PDF specification available free of charge in 1993, PDF remained a proprietary format, controlled by Adobe, until it was officially released as an open standard on July 1, 2008.  

Now, most (all?) \LaTeX{} interfaces come with {\tt pdflatex}, which simply creates a .pdf from a .tex file directly.

This is meaningful for you as it determines what file format you need your figures in.  If you use the original specification, your figures must be .eps (the `e' stands for encapsulated, which essentially has additional header information that gives a bounding
box for the figure).  Alternatively, if you use {\tt pdflatex}, the figures must be .pdf.  Note: I'm aware that other file formats can
be read in (e.g. .png, .jpg, ...).  Don't use these as they are not vector graphics and hence can't be scaled.

\textbf{Conclusion:} The take-away message is to use .pdf figures and {\tt pdflatex}.  Some stodgier journals still require you
to use .eps figures, though, so be careful what you're doing from the beginning so you don't have to regenerate a bunch of 
things (as an aside, you can always go .ps to .pdf.  The opposite can be trickier). 

\section{Referencing}
A crucial part of any scientific document will be referring to previous or forthcoming parts in the paper.  This could of course be
hard coded in, but that is almost surely going to cause problems at some point.  Luckily, \LaTeX{}allow for self-consistent and 
automatic referencing.  To wit:

\begin{verbatim}
\begin{equation}
Y = \mathbb{X}\beta + \sigma Z
\label{eq:regression}
\end{equation}
As we can see in equation \eqref{eq:regression}, ...
\end{verbatim}
\begin{equation}
Y = \mathbb{X}\beta + \sigma Z
\label{eq:regression}
\end{equation}
As we can see in equation \eqref{eq:regression}, ...

Now, if we move this equation or add ones before it, we will always be referencing the correct equation.  As an aside, \LaTeX{}
actually compiles parts of the document at a time.  So, it is common to see the following instead:

as we can see in equation (?), ...

Simply compile the document another time and this should go away (this is because \LaTeX{} first builds an index of references and then actually adds the references after a second compilation).  If not, check the output for references to 
non-existing labels.

Note that we can refernce anything that is an environment (theorems, equations, sections, tables, figures, ...).

\section{Functions and Redefining Commands}
Suppose we are writing a paper and we are continually writing 
\begin{verbatim}
\mathcal{G}
\end{verbatim}
which looks like $\mathcal{G}$.  Now, two things could happen.  The first is that a referree or coauthor could say `I don't like script
G for this, let's make it $\Omega$, instead.'  We could find and replace it, but what if we are using $\Omega$ for something else?
A second things is that we get sick of writing the whole thing out.  We can solve both of these problems at the same time.
Returning to the preamble
\begin{verbatim}
%Preamble
\documentclass{article}
\usepackage{subfig}
\usepackage{graphicx}
\usepackage{amsmath,amsthm,amssymb,amsfonts}
%Ignore for now%%% \usepackage[options]{natbib}
%Ignore for now%%% \renewcommand{\abstractname}{Not abstract}
\usepackage[colorlinks]{hyperref}

\newcommand{\G}{\mathcal{G}}
\end{verbatim}
Now, if we type 
\begin{verbatim}
$\G$
\end{verbatim}
we get $\G$.

Additonally, we can do more complicated things, as well.  What if we want to talk about a likelihood function at a fixed
data vector evaluated at a parameter, conditional on other parameters. 

Returning to the preamble
\begin{verbatim}
%Preamble
\documentclass{article}
\usepackage{subfig}
\usepackage{graphicx}
\usepackage{amsmath,amsthm,amssymb,amsfonts}
%Ignore for now%%% \usepackage[options]{natbib}
%Ignore for now%%% \renewcommand{\abstractname}{Not abstract}
\usepackage[colorlinks]{hyperref}

\newcommand{\G}{\mathcal{G}}
\renewcommand{\L}[3]{\mathcal{L}_{#1}(#2|#3)}
\end{verbatim}
This gives:
\begin{verbatim}
\L{Y}{\theta}{\eta}
\end{verbatim}
$\L{Y}{\theta}{\eta}$

\subsection{Using .sty files}
Often, you will end up with many such definitions.  Or, you will end up using the same series of definitions in multiple documents.  It is very useful to keep such information in a central file and import it.

Create a document called `exampleDefs.sty' and have it contain:

\begin{verbatim}
\newcommand{\G}{\mathcal{G}}
\renewcommand{\L}[3]{\mathcal{L}_{#1}(#2|#3)}
\end{verbatim}

Then, add this to your preamble
\begin{verbatim}
%Preamble
\documentclass{article}
\usepackage{subfig}
\usepackage{graphicx}
\usepackage{amsmath,amsthm,amssymb,amsfonts}
%Ignore for now%%% \usepackage[options]{natbib}
%Ignore for now%%% \renewcommand{\abstractname}{Not abstract}
\usepackage[colorlinks]{hyperref}

\usepackage{exampleDefs}
\end{verbatim}
\end{document}
